\documentclass[version=last, paper=A4, parskip=half, oneside,%
toc=bibliography, toc=listof, listof=leveldown]{scrbook}

% Fonts
\usepackage{fontspec}

% Languages
\usepackage[english]{babel}

% Microtype
\usepackage[babel, final]{microtype}

% Ragged text
\usepackage{ragged2e}

% Mathematics
\usepackage{mathtools}
\usepackage{amsthm}
\usepackage[algochapter, ruled, longend]{algorithm2e}
\usepackage{prftree}
\usepackage{unicode-math}

% Source code
\usepackage{fancyvrb}
\usepackage[newfloat]{minted}

% Boxes
\usepackage{varwidth}

% Tables
\usepackage{booktabs}
\usepackage{multirow}

% Lists
\usepackage[inline]{enumitem}

% Graphics
\usepackage{graphicx}

% Captioning
\usepackage[compatibility=false]{caption}
\usepackage{subcaption}

% Numbers
\usepackage[binary-units=true]{siunitx}
\usepackage{nth}

% Plotting
\usepackage{tikz}
\usepackage{tikzpeople}
\usepackage{pgfplots}
\usepackage{pgfplotstable}

% Quotations
\usepackage[strict, autopunct]{csquotes}

% Links and references
\usepackage{hyperref}
\usepackage[nameinlink]{cleveref}

% Bibliographies
\usepackage[hyperref, backref, backend=biber, block=space,%
loadfiles, giveninits, pagetracker]{biblatex}

% Dashes and hyphens
\usepackage[shortcuts]{extdash}

%
% Package setup
%

\defaultfontfeatures{Ligatures=TeX}

\addbibresource{\jobname.bib}

\unimathsetup{%
  math-style=ISO,
  bold-style=ISO%
}

\mathtoolsset{
  mathic%
}

\sisetup{%
  detect-all,
  detect-display-math,
  binary-units%
}

\hypersetup{%
  unicode,
  colorlinks,
  breaklinks,
  pdftitle={Error-Correcting Codes in the Rank Metric},
  pdfauthor={Dario Gjorgjevski},
  pdfsubject={Bachelor's Thesis},
  pdfkeywords={Coding Theory, Cryptography}%
}

\pgfplotsset{compat=newest}
\usetikzlibrary{calc, positioning, shapes}

\KOMAoptions{DIV=last}

% Lengths
\newlength{\savedcolsep}

% Theorems, lemmas, etc.
\theoremstyle{plain}
\newtheorem{theorem}{Theorem}
\newtheorem{lemma}{Lemma}
\newtheorem{corollary}{Corollary}
\newtheorem{proposition}{Proposition}

% Definitions
\theoremstyle{definition}
\newtheorem{definition}{Definition}

% Remarks and examples
\theoremstyle{remark}
\newtheorem*{remark}{Remark}
\newtheorem*{example}{Example}

% Custom lists
\newlist{steps}{enumerate}{1}
\setlist[steps, 1]{label={\emph{Step \arabic*}.}, ref={\arabic*}, leftmargin=*}
\crefname{stepsi}{step}{steps}
\Crefname{stepsi}{Step}{Steps}

\newlist{properties}{enumerate}{1}
\setlist[properties, 1]{label={\emph{Property \arabic*}.}, ref={\arabic*}, leftmargin=*}
\crefname{propertiesi}{property}{properties}
\Crefname{propertiesi}{Property}{Properties}

\newlist{conditions}{enumerate}{1}
\setlist[conditions, 1]{label={\emph{Condition \arabic*}.}, ref={\arabic*}, leftmargin=*}
\crefname{conditionsi}{condition}{conditions}
\Crefname{conditionsi}{Condition}{Conditions}

%
% Commands
%

% Notation from linear algebra
\AtBeginDocument{%
  \renewcommand*{\vec}{\symbf}
  \newcommand*{\mat}{\symbf}
  \newcommand*{\trans}{\top}%
}
\DeclareMathOperator{\rank}{rank}
\DeclareMathOperator{\lspan}{span}
\DeclareMathOperator{\supp}{supp}
\DeclarePairedDelimiter{\basis}{\langle}{\rangle}
\newcommand*{\M}{\ensuremath{\mathcal{M}}}
\newcommand*{\GL}{\ensuremath{\mathsf{GL}}}

% Notation from probability and statistics
\DeclareMathOperator{\E}{E}
\DeclareMathOperator{\var}{var}
\DeclareMathOperator{\cov}{cov}

% Notation from cryptography
\newcommand*{\pub}{\ensuremath{\mathsf{pub}}}
\newcommand*{\priv}{\ensuremath{\mathsf{priv}}}
\newcommand*{\enc}{\ensuremath{\mathsf{Enc}}}
\newcommand*{\dec}{\ensuremath{\mathsf{Dec}}}

% Standard sets
\newcommand*{\KK}{\ensuremath{\mathbb{K}}}
\newcommand*{\FF}{\ensuremath{\mathbb{F}}}
\newcommand*{\NN}{\ensuremath{\mathbb{N}}}
\newcommand*{\ZZ}{\ensuremath{\mathbb{Z}}}
\newcommand*{\RR}{\ensuremath{\mathbb{R}}}
\newcommand*{\CC}{\ensuremath{\mathbb{C}}}

% Complexity classes
\newcommand*{\NP}{\ensuremath{\mathcal{NP}}}

% Linear codes
\newcommand*{\Gop}{\ensuremath{\Gamma}}
\newcommand*{\Gab}{\ensuremath{\mathfrak{G}}}

% arg{min,max}
\DeclareMathOperator*{\argmin}{arg\,min}
\DeclareMathOperator*{\argmax}{arg\,max}

% Sampling
\newcommand*{\sample}{\ensuremath{\gets_{\mathrm{\$}}}}

% Computational problems
\newcommand*{\MDD}{\ensuremath{\mathsf{MDD}}}
\newcommand*{\CSD}{\ensuremath{\mathsf{CSD}}}
\newcommand*{\SW}{\ensuremath{\mathsf{SW}}}
\newcommand*{\MR}{\ensuremath{\mathsf{MR}}}

% Ceiling and floor
\DeclarePairedDelimiter{\floor}{\lfloor}{\rfloor}
\DeclarePairedDelimiter{\ceil}{\lceil}{\rceil}

% Absolute value and norm
\DeclarePairedDelimiter{\abs}{\lvert}{\rvert}
\DeclarePairedDelimiter{\norm}{\lVert}{\rVert}

% Custom norms and distances
\newcommand*{\normH}[1]{\ensuremath{\norm{#1}_{\mathrm{H}}}}
\newcommand*{\normR}[2]{\ensuremath{\norm{#1}_{#2}}}
\DeclareMathOperator{\dH}{d_H}
\DeclareMathOperator{\dR}{d_R}

% Set cardinality
\DeclarePairedDelimiter{\card}{\lvert}{\rvert}

% Gaussian binomial coefficient
\DeclareRobustCommand{\gbinom}{\genfrac[]{0pt}{}}

% Miscellaneous notation
\newcommand*{\sys}{\ensuremath{\mathsf{sys}}}
\newcommand*{\ext}{\ensuremath{\mathsf{ext}}}
\newcommand*{\reg}{\ensuremath{\mathsf{reg}}}

\newcommand*{\KS}{\ensuremath{\mathsf{KS}}}
\newcommand*{\GV}{\ensuremath{\mathsf{GV}}}
\newcommand*{\LB}{\ensuremath{\mathsf{LB}}}

\newcommand*{\PISD}{\textsc{Plain}\=/ISD}
\newcommand*{\LBISD}{LB\=/ISD}

%
% Title
%

\titlehead{\includegraphics{Logo.pdf}}
\subject{Bachelor's Thesis}
\title{Error\-/Correcting Codes in the Rank Metric}
\subtitle{With Applications to Cryptography}
\author{%
  Dario Gjorgjevski\\%
  \href{mailto:gjorgjevski.dario@students.finki.ukim.mk}{\nolinkurl{gjorgjevski.dario@students.finki.ukim.mk}}}
\date{\today}

%
% Document
%

\begin{document}

\addtotoclist[float]{loa}
\setuptoc{loa}{%
  chapteratlist, totoc, leveldown%
}

\newcommand*{\listofloaname}{\listalgorithmcfname}
\renewcommand{\listofalgorithms}{\listoftoc{loa}}

\frontmatter{}

\maketitle{}

\tableofcontents{}
\listoffigures{}
\listoftables{}
\listofalgorithms{}
\listoflistings{}

\addchap{Abstract}

Public\-/key cryptography was invented by Diffie and Hellman in order to remove
the need for trusted couriers when exchanging secret keys that will facilitate
secure communication.  The Diffie\--Hellman key exchange (DHKE) and the
Rivest\--Shamir\--Adleman (RSA) cryptosystem are two of the first public\-/key
constructions that remain largely unbroken and in use to this day.  However, the
publication of Shor's algorithm in 1994 meant that these algorithms are rendered
completely insecure given a quantum computer.  Even though quantum computing is
still far beyond reach, cryptographers are pushing for so\-/called
post\-/quantum algorithms, i.e., algorithms which are believed to be secure even
against attacks carried on quantum computers.  Error\-/correcting codes are a
popular way of building such algorithms and have been studied since roughly the
same time as RSA\@.  The McEliece cryptosystem is one such construction.  It is
based on the hardness of decoding random linear codes in the Hamming metric,
using a scrambled Goppa code that only the holder of some secret information can
decode efficiently.  Unfortunately, the size of this \enquote{secret
  information} can get quite high in practice.  Trying to alleviate this,
Gabidulin, Paramonov, and Tretjakov proposed an analogous cryptosystem that
utilizes error\-/correcting codes in the \emph{rank metric}: the GPT
cryptosystem.  The goal of this thesis is twofold: first, to provide a survey of
the GPT cryptosystem, Overbeck's structural attack on it, and the most recent
reparation by \textcite{Loi17}; and second, to investigate the applicability of
information set decoding in the rank metric.  Information set decoding (ISD) is
a family of algorithms that has been the most successful way to attack
McEliece\-/like cryptosystems, but has not been explored in the context of the
rank metric.

\mainmatter{}

\chapter{Introduction}\label{chap:introduction}

In its advent, cryptography was used with the intention of making the
communication between two parties secure in an ad\-/hoc manner.  During the late
\nth{20} century, the picture of cryptography radically changed.  A rich and
rigorous theory was developed allowing for formal arguments to be made regarding
the security of certain constructions.  Furthermore, the field of cryptography
now encompasses many well\-/defined objectives such as:
\begingroup
\setkomafont{labelinglabel}{\emph}
\setkomafont{labelingseparator}{\normalfont}
\begin{labeling}[~--]{non-repudiation}
\item[confidentiality] keeping information secret from all but those
  who are authorized to see it;
\item[data integrity] ensuring information has not been altered by unauthorized
  or unknown means;
\item[authentication] corroboration of the identity of an entity (e.g., a
  person, a computer terminal, a credit card, etc.); and
\item[non\-/repudiation] preventing the denial of previous commitments or
  actions.
\end{labeling}
\endgroup

Secure communication can be established in two settings: symmetric and
asymmetric.  As we mentioned, secure communication is concerned with the
\emph{confidentiality} of messages: we want to make sure that a person
intercepting a message that has been sent learns nothing (or \enquote{very
  little}) about the message's content.  To this end, \emph{encryption}
algorithms are employed.  We will review both the symmetric and asymmetric
settings, but the focus of the remainder of this thesis is going to be entirely
on algorithms for confidentiality in the asymmetric setting.

\paragraph{The Symmetric Setting} In the symmetric setting, two parties share
some secret information in advance\--- the \emph{symmetric key} (or just
\emph{key})\--- that they use whenever they wish to communicate secretly with
each other.  The party sending a message uses the key to \emph{encrypt} the
message, and the receiver uses the same key to \emph{decrypt} it upon receipt.
\Cref{fig:symmetric-setting} depicts typical communication in this setting.

\begin{figure}
  \centering
  \begin{tikzpicture}[node distance=3cm, auto]
    \node[alice] (alice) {Alice};
    \node[draw, right=of alice] (encryption) {\(\enc_k(m)\)} edge[<-] node[above] {Plaintext \(m\)} (alice);
    \node[draw, right=of encryption] (decryption) {\(\dec_k(c)\)} edge[<-] node[above] {Ciphertext \(c\)} (encryption);
    \node[bob, right=of decryption, mirrored] (bob) {Bob} edge[<-] node[above] {Plaintext \(m\)} (decryption);
    \coordinate (center) at ($(encryption)!0.5!(decryption)$);
    \node[draw, ellipse, below=1.5cm of center] (key) {Key \(k\)} edge[->] (encryption) edge[->] (decryption);
  \end{tikzpicture}
  \captionbelow{Communication in the symmetric
    setting}\label{fig:symmetric-setting}
\end{figure}

\paragraph{The Asymmetric Setting} The asymmetric setting aims to solve the
biggest problem of symmetric encryption: the need to securely exchange a key
beforehand.  Indeed, an observant person will ask: if two parties using
symmetric encryption could exchange a key without anyone knowing, why do they
not use the same means to exchange messages?  Historically, such key exchanges
had been done by \enquote{trusted couriers}, but in a network such as the
Internet this is anything but a feasible solution.  Cryptography in the
asymmetric setting is called \emph{public\-/key cryptography}.  Public\-/key
cryptography eliminates the need for trusted couriers by having users use pairs
of keys: \emph{public keys} which may be disseminated widely, and \emph{private
  keys} which are known only to their owners.

\Textcite{DH76} invented public\-/key cryptography by publishing the so\-/called
\emph{Diffie\--Hellman key exchange} (DHKE).  DHKE allows two parties to
\emph{generate} a shared secret in such a way that the secret cannot be deduced
by observing the communication.  A few years later, \textcite{RSA78} published
the RSA cryptosystem which can be used to encrypt and decrypt messages.  It
remains unbroken to this day and is one of the most widely used cryptosystems.
In a public\-/key cryptosystem such as RSA, any person can encrypt a message
using the public key of the intended receiver; but, once encrypted, the
ciphertext can be decrypted only with the receiver's private key.

\begin{displaycquote}{Sch96}
  Putting mail in the mailbox is analogous to encrypting with the public key;
  anyone can do it.  Just open the slot and drop it in.  Getting mail out of a
  mailbox is analogous to decrypting with the private key.  Generally it's hard;
  you need welding torches.  However, if you have the secret (the physical key
  to the mailbox), it's easy to get mail out of a mailbox.
\end{displaycquote}

Public\-/key cryptography revolves largely around the notion of \emph{one\-/way}
and \emph{trapdoor} functions.  Informally, a one\-/way function is a function
that is easy to compute on every input, but hard to invert given the image of a
random input; on the other hand, a trapdoor function is a one\-/way function
that can be inverted given some special information, called the
\enquote{trapdoor}.  The RSA cryptosystem relies on the hardness of integer
factorization to construct a trapdoor permutation for encryption.  The only
computationally feasible way to invert it is by knowing the trapdoor \--{} the
prime factors.  However, \textcite{Sho97} formulated an algorithm which can
factor integers efficiently and thus \enquote{break} RSA\@.  There is only one
drawback: to run it, one needs a quantum computer.  The discovery of Shor's
algorithm led researchers to consider \emph{post\-/quantum algorithms}:
public\-/key algorithms thought to be secure against attacks by quantum
computers.

Interestingly, a cryptosystem from roughly the same era as DHKE and RSA is
believed to be post\-/quantum: the \emph{McEliece cryptosystem}, first
formulated by \textcite{Eli78}.  The security of McEliece relies on the hardness
of decoding a random linear code.  Namely, a user publishes a scrambled Goppa
code as the public key and keeps the \enquote{unscrambled} code in private.
Senders simply encode the message with the public code and add some errors.
Attackers are faced with decoding the scrambled code which has no obvious
structure, while the intended recipient has all the necessary information to
unscramble and use (one of the) efficient decoding algorithms for Goppa codes.

The best generic attacks against the McEliece cryptosystem come from a family
of algorithms known as \emph{information set decoding}~\cite{B+12}.  The main
disadvantage of McEliece is its key size: to achieve \num{128}\=/bit security
against information set decoding, a key of around \SI{192}{\kilo\byte} is
required.  \Textcite{GPT91} proposed the GPT cryptosystem which uses
error\-/correcting codes in \emph{rank metric}.  (The Goppa codes used in
McEliece correct errors in the \emph{Hamming metric}.)  GPT is thought to
require considerably smaller keys for equivalent security.  The goal of this
thesis is to provide a survey of GPT and formulate the information set decoding
family of algorithms in the rank metric.

\chapter{The GPT Cryptosystem}\label{chap:GPT}

\section{Foundations}

Code\-/based cryptography is concerned with the use of error\-/correcting codes
for cryptographic purposes.  Much of the motivation and background for the
theory comes from the landmark work of \textcite{Sha48}.  In it, he laid the
foundations of \emph{information theory}, which deals with the transmission of
messages over noisy channels.  A typical model of information transmission is
given in \cref{fig:information-transmission-model}.

\begin{figure}
  \centering
  \begin{tikzpicture}[auto]
    \node[draw] (source) {Source};
    \node[draw, right=of source] (transmitter) {Transmitter} edge[<-] (source);
    \node[draw, right=of transmitter] (channel) {Channel} edge[<-] (transmitter);
    \node[draw, right=of channel] (receiver) {Receiver} edge[<-] (channel);
    \node[draw, right=of receiver] (destination) {Destination} edge[<-] (receiver);

    \node[draw, ellipse, below=of channel] (noise) {Noise source} edge[->] (channel);

    \coordinate (source-transmitter) at ($(source.east)!0.5!(transmitter.west)$);
    \coordinate (transmitter-channel) at ($(transmitter.east)!0.5!(channel.west)$);
    \coordinate (channel-receiver) at ($(channel.east)!0.5!(receiver.west)$);
    \coordinate (receiver-destination) at ($(receiver.east)!0.5!(destination.west)$);

    \node[above=of source-transmitter] (message-1) {Message} edge[dashed] (source-transmitter);
    \node[above=of transmitter-channel] (signal) {Signal} edge[dashed] (transmitter-channel);
    \node[above=of channel-receiver] (noisy-signal) {Noisy signal} edge[dashed] (channel-receiver);
    \node[above=of receiver-destination] (message-2) {Message} edge[dashed] (receiver-destination);
  \end{tikzpicture}
  \captionbelow{Transmitting information over a noisy
    channel}\label{fig:information-transmission-model}
\end{figure}

We assume that a sequence of symbols from a finite alphabet\--- the
\emph{message}\--- is transmitted over a noisy channel.  (The noise is usually
assumed to be additive.)  Each symbol of the message has a probability to be
affected by error; in a sense that instead of the symbol \(\mathtt{a}\), the
receiver sees the symbol \(\mathtt{b}\).  To overcome this problem, the
transmitted information will not only contain the message, but will also include
redundancy based on the message's contents.

Specifically, each message is mapped injectively into a \emph{codeword} through
a process called \emph{encoding}.  Instead of the message, the codeword is
transmitted over the noisy channel.  The receiver then uses a \emph{decoding}
algorithm to detect and possibly correct any errors that might have occurred
during the transmission.  In cryptography, the transmission is assumed to be
error\-/free, but noise is added deliberately to the codeword as part of the
encryption process.

\section{Linear Codes}

\Cref{sec:LC-basic-concepts} defines linear codes and information sets, while
\cref{sec:LC-computational-problems} provides an overview and briefly discusses
the complexity of solving some computational problems related to the decoding of
random linear codes.

\subsection{Basic Concepts}\label{sec:LC-basic-concepts}

\begin{definition}[Linear code]
  An \([n, k]\)\=/code \(\mathcal{C}\) over a finite field \(\FF\) is a
  \(k\)\=/dimensional subspace of the vector space \(\FF^n\).  \(\mathcal{C}\)
  is an \([n, k, d]\)\=/code if
  \(d = \min_{\vec{x}, \vec{y} \in \mathcal{C}} \norm{\vec{x} - \vec{y}}\) for
  some norm \(\norm{\cdot}\).  The elements of \(\mathcal{C}\) are called
  \emph{codewords}.  \(d\) is called the \emph{minimum distance} of the code
  with respect to \(\norm{\cdot}\).
\end{definition}

\begin{definition}[Generator matrix]
  A full\-/rank matrix \(\mat{G} \in \FF^{k \times n}\) is said to be a
  \emph{generator matrix} for the \([n, k]\)\=/code \(\mathcal{C}\) if its rows
  span \(\mathcal{C}\) over \(\FF\).  In other words,
  \[
    \mathcal{C} = \{\vec{x} \mat{G} : \vec{x} \in \FF^k\}\text.
  \]
  \(\mat{G}\) defines an \emph{encoding map}
  \(f_{\mat{G}}\colon \FF^k \to \FF^n\) given by
  \(\vec{x} \mapsto \vec{x} \vec{G}\).
\end{definition}

Before defining the parity\-/check matrix, we introduce the dual of a code.

\begin{definition}[Dual]
  Let \(\mathcal{C}\) be an \([n, k]\)\=/code.  The \emph{dual} of
  \(\mathcal{C}\) is the \([n, n - k]\)\=/code \(\mathcal{C}^{\perp}\) defined
  by:
  \[
    \mathcal{C}^{\perp} \coloneqq
    \{\vec{y} \in \FF^n : \forall \vec{x} \in \mathcal{C}\ldotp \vec{x} \vec{y}^{\trans} = 0\}\text.
  \]
\end{definition}

\begin{definition}[Parity\-/check matrix]
  A \emph{parity\-/check} matrix of a code \(\mathcal{C}\) is a generator matrix
  of its dual, \(\mathcal{C}^{\perp}\).
\end{definition}

Let \(\mat{G}\) be a generator matrix of an \([n, k]\)\=/code \(\mathcal{C}\)
and \(\mat{H}\) be a parity\-/check matrix of \(\mathcal{C}\).  Then
\(\mat{G} \mat{H}^{\trans} = \vec{0}\).  Conversely, any
\(\mat{H} \in \FF^{(n - k) \times n}\) that satisfies
\(\mat{G} \mat{H}^{\trans} = \vec{0}\) is a parity\-/check matrix of
\(\mathcal{C}\).

A code \(\mathcal{C}\) has many generator matrix representations.  \(\mat{G}\)
is said to be in \emph{systematic form} if its first \(k\) columns form the
\(k \times k\) identity matrix, \(\mat{I}_k\).  For any systematic generator
matrix \(\mat{G}_{\sys}\), each entry of the message vector appears among the
entries of the codeword.  Given the generator matrix in systematic form
\(\mat{G}_{\sys} = \begin{bmatrix} \mat{I}_k & \mat{Q} \end{bmatrix}\), the
parity\-/check matrix in \emph{canonical form} is
\(\mat{H} = \begin{bmatrix} -\mat{Q}^{\trans} & \mat{I}_{n - k} \end{bmatrix}\).

\begin{example}
  Consider the matrix \(\mat{G} \in \FF_2^{4 \times 7}\):
  \[
    \mat{G} =
    \begin{bmatrix}
      1 & 0 & 0 & 0 & 1 & 1 & 1 \\
      0 & 1 & 0 & 0 & 0 & 1 & 1 \\
      0 & 0 & 1 & 0 & 1 & 0 & 1 \\
      0 & 0 & 0 & 1 & 1 & 1 & 0
    \end{bmatrix}\text.
  \]
  We have that \(\rank \mat{G} = 4\), so \(\mat{G}\) generates a
  \([7, 4]\)\=/code over \(\FF_2\).  Let \(\mat{H}\) be given as
  \[
    \mat{H} =
    \begin{bmatrix}
      1 & 0 & 1 & 1 & 1 & 0 & 0 \\
      1 & 1 & 0 & 1 & 0 & 1 & 0 \\
      1 & 1 & 1 & 0 & 0 & 0 & 1
    \end{bmatrix}\text.
  \]
  We have \(\mat{G} \mat{H}^{\trans} = \vec{0}\) and \(\rank \mat{H} = 3\), so
  \(\mat{H}\) is a parity\-/check matrix of the code.
\end{example}

\begin{definition}[Syndrome]
  Let \(\mat{H}\) be a parity\-/check matrix for \(\mathcal{C}\).  The
  \emph{syndrome} of a vector \(\vec{y} \in \FF^n\) with respect to \(\mat{H}\)
  is the (column) vector \(\mat{H} \vec{y}^{\trans} \in \FF^{n - k}\).
\end{definition}

\begin{definition}[Information set]\label{def:information-set}
  An \emph{information set} of an \([n, k]\)\=/code \(\mathcal{C}\) with
  generator matrix \(\mat{G}\) is a size\=/\(k\) subset
  \(\mathcal{I} \subset \{1, \ldots, n\}\) such that the columns of \(\mat{G}\)
  indexed by \(\mathcal{I}\) form an invertible \(k \times k\) submatrix,
  \(\mat{G}_{\mathcal{I}}\).  In particular,
  \(\mat{G}_{\mathcal{I}}^{-1} \mat{G}\) is a generator matrix of
  \(\mathcal{C}\) in systematic form (up to a permutation of columns).  If
  \(\mat{H}\) is a parity\-/check matrix, then the columns indexed by
  \(\{1, \ldots, n\} \smallsetminus \mathcal{I}\) form an invertible
  \((n - k) \times (n - k)\) matrix.
\end{definition}

Let \(\vec{c} \coloneqq \vec{x} \mat{G}_{\mathcal{I}}^{-1} \mat{G}\) for some
vector \(\vec{x} \in \FF^k\).  The \(\mathcal{I}\)\=/indexed entries of
\(\vec{c}\) form the information vector \(\vec{x}\).  These entries are called
\emph{information symbols}; the rest of the entries are called
\emph{parity\-/check symbols}.

\subsection{Computational Problems}\label{sec:LC-computational-problems}

The decoding of random codes lends itself to many computational problems.  The
\emph{general decoding problem} can be stated as follows: given an encoding
function \(f\colon \FF^k \to \FF^n\) and an encoded message \(f(\vec{x})\) that
is inflicted with random noise, determine the original message \(\vec{x}\).
Here, \(f\) need not be linear and the noise need not be additive.

However, for practical reasons, we restrict ourselves only to the linear
setting, in which the encoding encoding function is specified by the generator
matrix \(\mat{G}\) and generates a codeword
\[
  \vec{c} \coloneqq f_{\mat{G}}(\vec{x}) = \vec{x} \mat{G}\text.
\]
The codeword is inflicted with additive noise and received as
\(\vec{y} \coloneqq \vec{c} + \vec{e}\), where \(\vec{e} \in \FF^n\) denotes the
noise.

\begin{definition}[\textsc{Minimum Distance Decoding}]
  Let \(\mathcal{C}\) be an \([n, k, d]\)\=/code.  Given a received word
  \(\vec{y}\) and \(w \in \NN\), find a codeword \(\vec{c} \in \mathcal{C}\)
  such that \(\norm{\vec{y} - \vec{c}} \le w\).  If no such codeword exists,
  output \(\perp\).  We denote the set of all solutions to this problem by
  \(\MDD(\mathcal{C}, \vec{y}, w)\).
\end{definition}

When \(\norm{\cdot}\) is the Hamming norm, the decision version of
\textsc{Minimum Distance Decoding} was proven \NP\=/complete for binary codes by
\textcite{BEvT78} and in the general case by \textcite{Bar94}.  \textsc{Minimum
  Distance Decoding} can also be formulated as a problem of syndrome decoding.

\begin{definition}[\textsc{Computational Syndrome Decoding}]
  Let \(\mathcal{C}\) be an \([n, k, d]\)\=/code with parity\-/check matrix
  \(\mat{H}\).  Given a syndrome \(\vec{s} \in \FF^{n - k}\) and \(w \in \NN\),
  find a word \(\vec{e} \in \FF^n\) such that
  \(\mat{H} \vec{e}^{\trans} = \vec{s}\) and \(\norm{\vec{e}} \le w\).  If no
  such word exists, output \(\perp\).  We denote the set of all solutions to
  this problem by \(\CSD(\mat{H}, \vec{s}, w)\).
\end{definition}

\begin{figure}
  \centering
  \begin{tikzpicture}[auto]
    \draw[-latex] (-0.5, 0) -- (10.5, 0);
    \foreach \x in {0, 4} \draw[shift={(\x, 0)}] (0, 3pt) -- (0, -3pt);
    \node (w) at (11, 0) {\(w\)};
    \node (zero) at (0, -0.5) {\(0\)};
    \node (GV) at (4, -0.5) {\(\tau_{\GV}\)};
    \draw[-latex] (3.5, 0.5) -- node[midway, above, label={\strut\emph{at most one solution}}] {} (0.5, 0.5);
    \draw[-latex] (4.5, 0.5) -- node[midway, above, label={\strut\emph{multiple solutions}}] {} (7.5, 0.5);
  \end{tikzpicture}
  \captionbelow{Number of solutions in
    \(\CSD(\mat{H}, \vec{s}, w)\)}\label{fig:CSD-cardinality}
\end{figure}

If we fix \(n\) and \(k\), the number of solutions in
\(\CSD(\mat{H}, \vec{s}, w)\) is illustrated in \cref{fig:CSD-cardinality}.
\(\tau_{\mathsf{GV}}\) is the \emph{Gilbert\--Varshamov radius}: the maximum
radius of spheres centered at codewords of \(\mathcal{C}\) so that they together
fit in \(\FF^n\).  For the purposes of cryptanalysis, we will restrict ourselves
to the case when \(\CSD(\mat{H}, \vec{s}, w) \ne \emptyset\) and
\(w < \tau_{\mathsf{GV}}\); i.e., the case with \emph{exactly one} solution.
Closely related is the problem of finding a small\-/weight codeword
\(\vec{c} \in \mathcal{C}\).

\begin{definition}[\textsc{Subspace Weights}]
  Let \(\mathcal{C}\) be an \([n, k, d]\)\=/code.  Given \(w \in \NN\), find a
  codeword \(\vec{c} \in \mathcal{C}\) such that \(\norm{\vec{c}} \le w\).  If
  no such codeword exists, output \(\perp\).  We denote the set of all solutions
  to this problem by \(\SW(\mathcal{C}, w)\).
\end{definition}

To see the importance of \textsc{Subspace Weights}, let
\(\vec{y} = \vec{c} + \vec{e}\) be a received word, and consider
\(\mathcal{C}'\), the code generated by
\[
  \mat{G}' \coloneqq \begin{bmatrix} \mat{G} \\ \vec{y} \end{bmatrix}\text.
\]
In order for \(\vec{c}\) to be uniquely decodable,
\(\norm{\vec{e}} \le t < \floor{(d - 1) / {2}}\) must hold, where \(t\) is the
error\-/correcting capacity of \(\mathcal{C}\).  Hence
\[
  \mathcal{C}' = \mathcal{C} \cup \{\vec{y} + \mathcal{C}\}
\]
has minimum distance \(d' = \norm{\vec{e}}\) and therefore the codeword of
minimum weight is \(\vec{e}\).  Solving \(\SW(\mathcal{C}', w')\) for
\(w' \in \{0, \ldots, t\}\) will let us find \(\vec{e}\).

\section{The McEliece Cryptosystem}

This section describes the McEliece cryptosystem: its parameters, key
generation, and encryption and decryption procedures.

\paragraph{System parameters} Choose \(k, n \in \NN\) such that \(k < n\); as
usual, \(n\) is the length and \(k\) is the dimension of the code.  Choose
\(t \in \NN\) to be the desired error\-/correcting capability.  Recall that
\(\Gop\) corrects errors in the Hamming metric.

\paragraph{Key generation} Generate a random classical Goppa code
\(\Gop = \Gop_q(a_1, \ldots, a_n; g)\) of dimension \(k\) over \(\FF_q\) with
an error\-/correcting capability of \(w\).  Choose a random \(n \times n\)
permutation matrix \(\mat{P}\) and an invertible \(k \times k\) matrix
\(\mat{S}\).

\paragraph{Public key} Tuple \((\hat{\mat{G}}, n, k, w)\), where
\(\hat{\mat{G}} \coloneqq \mat{S} \mat{G} \mat{P}\).  \(\mat{G}\) is a
generator matrix of \(\Gop\).

\paragraph{Private key} Tuple \((\Gop, \mat{G}, \mat{P}, \mat{S})\).

nformation needs to be embedded in a length\=/\(k\) word
\(\vec{x} \in \FF_q^k\) in order to be suitable for the encryption algorithm.
Then, \(\vec{x}\) can be encrypted with \cref{alg:mceliece-encryption}.  The
legitimate owner of the private key can use \cref{alg:mceliece-decryption} to
decode a ciphertext \(\vec{y}\).

\begin{algorithm}
  \caption{McEliece encryption}\label{alg:mceliece-encryption}
  \DontPrintSemicolon{}%
  \KwIn{Message \(\vec{x} \in \FF_q^k\), public generator matrix
    \(\hat{\mat{G}} \in \FF_q^{k \times n}\), and \(w \in \NN\)}%
  \KwOut{Ciphertext \(\vec{y} \in \FF_q^n\)}%
  \Begin{%
    \(\vec{c} \gets \vec{x} \hat{\mat{G}}\)\;%
    \(\vec{e} \sample \FF_q^n\) with \(\normH{\vec{e}} = w\)\;%
    \Return{\(\vec{y} = \vec{c} + \vec{e}\)}\;%
  }
\end{algorithm}

\begin{algorithm}
  \caption{McEliece decryption}\label{alg:mceliece-decryption}
  \DontPrintSemicolon{}%
  \KwIn{Ciphertext \(\vec{y} \in \FF_q^n\) and private key
    \((\Gop, \mat{G}, \mat{P}, \mat{S})\)}%
  \KwOut{Message \(\vec{x} \in \FF_q^k\)}%
  \Begin{%
    \(\vec{y} \mat{P}^{-1} = \vec{x} \mat{S} \mat{G} + \vec{e} \mat{P}^{-1}\)%
    \tcc*[r]{note that \(\normH{\vec{e} \mat{P}^{-1}} = w\)}
    \(\vec{x} \gets\) decode \(\vec{y} \mat{P}^{-1}\) using an efficient
    decoding algorithm for \(\Gop\)\;%
    \Return{\(\vec{x}\)}\;%
  }
\end{algorithm}

\section{Gabidulin Codes}

Gabidulin codes~\cite{Gab85} correct errors in the rank metric (also called
pattern errors) efficiently, which in general is harder than correcting errors
in the Hamming metric.  \Textcite{GPT91} proposed a cryptosystem based on
Gabidulin codes, known as GPT\@.  Because of its better resistance against
general decoding attacks, smaller key sizes were proposed for GPT compared to
the McEliece cryptosystem.  In this section, we define Gabidulin codes and give
a formal definition of the GPT cryptosystem.

\subsection{The Rank Metric}

Codes in the rank metric (e.g., Gabidulin codes) are linear codes over the
finite field \(\FF_{q^m}\) for \(q\) (power of a) prime and \(m \in \NN\).  As
suggested by their name, they use a special concept of distance.

\begin{definition}[Rank norm]
  Let
  \(\vec{x} \coloneqq \begin{bmatrix} \vec{x}_1 & \cdots & \vec{x}_n \end{bmatrix} \in \FF_{q^m}^n\)
  and \(\{\beta_1, \ldots, \beta_m\}\) be a basis of \(\FF_{q^m}\) over
  \(\FF_q\).  We can write
  \begin{equation}\label{eq:extension-field-as-vector-space}
    x_i = \sum_{j = 1}^m x_{i, j} \beta_j
  \end{equation}
  for all \(i = 1, \ldots, n\); with \(x_{i, j} \in \FF_q\).  The \emph{rank
    norm} \(\normR{\cdot}{q}\) is defined as
  \[
    \normR{\vec{x}}{q} \coloneqq
    \rank\left(
      \begin{bmatrix}
        x_{1, 1} & \cdots & x_{1, m} \\
        \vdots & \ddots & \vdots \\
        x_{n, 1} & \cdots & x_{n, m}
      \end{bmatrix}
    \right)\text,
  \]
  with the \(x_{i, j}\)'s as given by \cref{eq:extension-field-as-vector-space}.
\end{definition}

The rank norm of \(\vec{x} \in \FF_{q^m}^n\) is independent of the choice of
basis and induces a metric called the \emph{rank metric} defined in
\cref{def:rank-metric}.  Note that every \(\mat{T} \in \GL_n(\FF_q)\) is an
isometry for the rank metric:
\(\normR{\vec{x} \mat{T}}{q} = \normR{\vec{x}}{q}\).

\begin{definition}[Rank metric]\label{def:rank-metric}
  Let \(\vec{x}, \vec{y} \in \FF_{q^m}^n\).  The function
  \(\dR(\vec{x}, \vec{y}) \coloneqq \normR{\vec{x} - \vec{y}}{q}\) is a metric
  over \(\FF_{q^m}^n\), called the \emph{rank metric} (or \emph{rank distance}).
\end{definition}

\begin{lemma}
  Let \(\normH{\cdot}\) denote the Hamming norm.  Then, for all
  \(\vec{x} \in \FF_{q^m}^n\),
  \[
    \normR{\vec{x}}{q} \le \normH{\vec{x}}\text.
  \]
  \begin{proof}
    Let \(\normH{\vec{x}} \eqqcolon k\).  By definition, \(\vec{x}\) has \(k\)
    nonzero entries.  Thus, the remaining \(n - k\) entries are all zero and do
    not influence \(\normR{\vec{x}}{q}\) whatsoever.  The result immediately
    follows.  Equality holds if and only if all \(k\) nonzero entries are
    linearly independent over \(\FF_q\).
  \end{proof}
\end{lemma}

\begin{lemma}[Singleton bound]
  Let \(\mathcal{C}\) be an \([n, k, d]\)\=/code over \(\FF_q\) in the Hamming
  metric.  Then, the \emph{Singleton bound} states that
  \[
    q^k \le q^{n - d + 1} \implies k \le n - d + 1\text,
  \]
  which is usually written as
  \[
    d \le n - k + 1\text.
  \]
  \begin{proof}
    See~\cite{Sin64}.
  \end{proof}
\end{lemma}

Due to the Singleton bound in the Hamming metric, the minimum rank distance
\(d\) of an \([n, k, d]\)\=/code over \(\FF_{q^m}\) satisfies
\begin{equation}\label{eq:singleton-bound}
  d \le n - k + 1\text.
\end{equation}
An alternative bound\--- tighter than that in \cref{eq:singleton-bound} when
\(n \le m\)\--- is given in~\cite{Gab85}:
\[
  d \le \frac{m}{n}(n - k) + 1\text.
\]
Combining the two bounds gives
\[
  d \le \min\{m, n\} - \max\{m, n\} k + 1\text.
\]

\subsection{Generating Gabidulin Codes}

Codes which achieve the Singleton bound in the rank metric with equality are
called Maximum Rank Distance (MRD) codes.  \Textcite{Gab85} constructed a family
of MRD codes over \(\FF_{q^m}\) of length \(n \le m\).

For any \(x \in \FF_{q^m}\) and any \(i \in \ZZ\), the quantity \(x^{q^i}\) is
denoted by \(x^{[i]}\).  The notation is extended to vectors and matrices in a
component\-/wise manner.  \Cref{lem:frobenius-automorphism-properties} shows
some useful properties of the operation.

\begin{lemma}\label{lem:frobenius-automorphism-properties}
  For any \(\mat{A} \in \FF_{q^m}^{l \times s}\) and
  \(\mat{B} \in \FF_{q^m}^{k \times n}\), and for any
  \(\alpha, \beta \in \FF_q\):
  \begin{enumerate}
  \item If \(l = k\) and \(s = n\), then
    \[
      {(\alpha \mat{A} + \beta \mat{B})}^{[i]} = \alpha \mat{A}^{[i]} + \beta \mat{B}^{[i]}\text.
    \]
  \item If \(s = k\) then
    \[
      {(\mat{A} \mat{B})}^{[i]} = \mat{A}^{[i]} \mat{B}^{[i]}\text.
    \]
    In particular, if \(\mat{S} \in \GL_n(\FF_{q^m})\), then
    \(\mat{S}^{[i]} \in \GL_n(\FF_{q^m})\) and
    \[
      {(\mat{S}^{[i]})}^{-1} = {(\mat{S}^{-1})}^{[i]}\text.
    \]
  \end{enumerate}
  \begin{proof}
    The proof of the two points comes directly from the properties of the
    Frobenius operators: multiplicative and \(\FF_q\)\=/linear.  For the last
    point, note that since \(\mat{S} \mat{S}^{-1} = \mat{I}_n\), it is also true
    that \(\mat{S}^{[i]} {(\mat{S}^{-1})}^{[i]} = \mat{I}_n\).  This implies
    that \(\mat{S}^{[i]} \in \GL_n(\FF_{q^m})\) and hence
    \({(\mat{S}^{[i]})}^{-1} = {(\mat{S}^{-1})}^{[i]}\).
  \end{proof}
\end{lemma}

We are now ready to introduce the Gabidulin codes.

\begin{definition}[Gabidulin code]
  Let
  \(\vec{g} \coloneqq \begin{bmatrix} g_1 & \cdots & g_n \end{bmatrix} \in \FF_{q^m}^n\)
  with all \(g_i\) independent over the base field.  (This implies \(n \le m\).)
  The code spanned by
  \begin{equation}\label{eq:gabidulin-code-generator}
    \mat{G} \coloneqq
    \begin{bmatrix}
      g_1 & \cdots & g_n \\
      g_1^{[1]} & \cdots & g_n^{[1]} \\
      \vdots & \ddots & \vdots \\
      g_1^{[k - 1]} & \cdots & g_n^{[k - 1]}
    \end{bmatrix}
  \end{equation}
  is called a \emph{Gabidulin code} with dimension \(k\).  The vector
  \(\vec{g}\) is called a \emph{generator vector} of the Gabidulin code.  We
  denote this code \(\Gab_k(\vec{g})\).  A matrix of the form
  \cref{eq:gabidulin-code-generator} is called a \emph{\(q\)\=/Vandermonde
    matrix}.
  \begin{remark}
    The vector \(\vec{g}\) is not unique; all vectors of the form
    \(\gamma \vec{g}\) with \(\gamma \in \FF_{q^m} \smallsetminus \{0\}\)
    generate the same code.
  \end{remark}
\end{definition}

\begin{theorem}
  Gabidulin codes are Maximum Rank Distance (MRD) codes.
  \begin{proof}
    It is sufficient to show that for any \(\mat{X} \in \FF_{q}^{k \times n}\)
    with \(\rank \mat{X} = k\), \(\rank(\mat{G} \mat{X}^{\trans}) = k\) as well.
    Note that
    \[
      \mat{G} \mat{X}^{\trans} =
      \begin{bmatrix}
        f_1 & \cdots & f_n \\
        f_1^{[1]} & \cdots & f_n^{[1]} \\
        \vdots & \ddots & \vdots \\
        f_1^{[k - 1]} & \cdots & f_n^{[k - 1]}
      \end{bmatrix}
    \]
    with
    \[
      \begin{bmatrix}
        f_1 & \cdots & f_k
      \end{bmatrix} =
      \begin{bmatrix}
        g_1 & \cdots & g_k
      \end{bmatrix}\mat{X}^{\trans}\text.
    \]
    Since \(\normR{g}{q} = n\), \(\normR{f}{q} = \min\{n, \rank\mat{X}\} = k\).
    From here, \(\rank(\mat{G} \mat{X}^{\trans}) = k\).
\end{proof}
\end{theorem}

An \([n, k]\)\=/Gabidulin code has minimum distance \(d = n - k + 1\) and
corrects errors of rank up to \(t \coloneqq \floor{(n - k) / {2}}\).  For all
\(\mat{T} \in \operatorname{GL}_n(\FF_q)\), \(\mat{G} \mat{T}\) is a generator
matrix of the Gabidulin code with generator vector \(\vec{g} \mat{T}\).

\begin{lemma}
  Let \(\Gab_k(\vec{g})\) be an \([n, k]\)\=/Gabidulin code over \(\FF_{q^m}\)
  with generator vector \(\vec{g}\).  The dual of \(\Gab_k(\vec{g})\) is the
  Gabidulin code \(\Gab_{n - k}(\vec{h})\), where
  \(\vec{h} = \vec{y}^{[-(n - k - 1)]}\) for some
  \(\vec{y} \in {\Gab_{n - 1}(\vec{g})}^{\perp}\).
  \begin{proof}
    Remark from \cref{eq:gabidulin-code-generator} that
    \[
      {\Gab_{n - 1}(\vec{g})}^{\perp} \subset
      {\Gab_{n - 2}(\vec{g})}^{\perp} \subset \cdots \subset
      {\Gab_{k}(\vec{g})}^{\perp}\text.
    \]
    Since \({\Gab_{n - 1}(\vec{g})}^{\perp}\) is an MRD code of dimension \(1\),
    its minimum distance is \(d = n\).  Thus, all of its nonzero elements are of
    rank \(n\).  Let \(\vec{y} \in {\Gab_{n - 1}(\vec{g})}^{\perp}\) with
    \(\vec{y} \ne \vec{0}\).  We have
    \[
      \forall i \in \{0, \ldots, n - 2\}\ldotp
      \sum_{j = 1}^n y_j g_j^{[i]} = 0\text.
    \]
    This implies that
    \[
      \forall i \in \{0, \ldots, n - 2\}\ldotp
      \sum_{j = 1}^n y_j^{[-1]} g_j^{[i - 1]} = 0\text,
    \]
    and in particular,
    \[
      \forall i \in \{0, \ldots, n - 3\}\ldotp
      \sum_{j = 1}^n y_j^{[-1]} g_j^{[i - 1]} = 0\text.
    \]
    Thus \(\vec{y}^{[-1]} \in {\Gab_{n - 2}(\vec{g})}^{\perp}\).  We can argue
    by induction that
    \[
      \forall u \in \{0, \ldots, n - 1\}\ldotp
      \vec{y}^{[-u]} \in {\Gab_{n - 1 - u}(\vec{g})}^{\perp}\text,
    \]
    and for a given \(u \in \{0, \ldots, n - 1\}\) we have
    \[
      \forall i \in \{0, \ldots, u\}\ldotp
      \vec{y}^{[-u + i]} \in {\Gab_{n - 1 - u + i}(\vec{g})}^{\perp} \subset
      {\Gab_{n - 1 - u}(\vec{g})}^{\perp}\text.
    \]
    Notably, for \(u = n - k - 1\) and \(\vec{h} \coloneqq \vec{y}^{[-u]}\), we
    have \(\vec{h}^{[i]} \in {\Gab_k(\vec{g})}^{\perp}\) for all
    \(i \in \{0, \ldots, n - k - 1\}\).  In other words,
    \[
      {\Gab_k(\vec{g})}^{\perp} = \Gab_{n - k}(\vec{g})\text.
    \]
  \end{proof}
\end{lemma}

\begin{corollary}
  The parity\-/check matrix of \(\Gab_k(\vec{g})\) can be written as
  \[
    \mat{H} =
    \begin{bmatrix}
      h_1 & \cdots & h_n \\
      h_1^{[1]} & \cdots & h_n^{[1]} \\
      \vdots & \ddots & \vdots \\
      h_1^{[d - 2]} & \cdots & h_n^{[d - 2]}
    \end{bmatrix}\text,
  \]
  for an appropriate choice of \(\vec{h}\).  (Recall that in the case of
  Gabidulin codes, \(d = n - k + 1\).)
\end{corollary}

\begin{example}
  Let \(\FF_2^2 = \FF_2[\alpha]\) and
  \[
    \mat{G} = \begin{bmatrix} 1 & \alpha \end{bmatrix}\text.
  \]
  Then the code generated by \(\mat{G}\) is
  \[
    \begin{aligned}
      \mathcal{C} &=
      \left\{
        \begin{bmatrix} 0 & 0 \end{bmatrix},
        \begin{bmatrix} 1 & \alpha \end{bmatrix},
        \begin{bmatrix} \alpha & \alpha^2 \end{bmatrix},
        \begin{bmatrix} \alpha^2 & 1 \end{bmatrix}
      \right\} \\
      &\cong
      \left\{
        \begin{bmatrix} 0 & 0 \\ 0 & 0 \end{bmatrix},
        \begin{bmatrix} 1 & 0 \\ 0 & 1 \end{bmatrix},
        \begin{bmatrix} 0 & 1 \\ 1 & 1 \end{bmatrix},
        \begin{bmatrix} 1 & 1 \\ 1 & 0 \end{bmatrix}
      \right\}
    \end{aligned}\text.
  \]
  It has dimension \(1\) (over \(\FF_{2^2}\)) and minimum distance \(2\) (over
  \(\FF_2\)).  A respective parity\-/check matrix is
  \[
    \mat{H} = \begin{bmatrix} \alpha & 1 \end{bmatrix}\text.
  \]
\end{example}

\section{Decoding Gabidulin Codes}\label{sec:gabidulin-code-decoding}

\begin{definition}[Linearized polynomial]
  A \emph{linearized polynomial} \(F(z)\) over \(\FF_{q^m}\) is a polynomial of
  the form \(F(z) = \sum_{i = 0}^u f_i z^{[i]}\), where \(f_i \in \FF_{q^m}\)
  for all \(0 \le i \le u\).  We refer to \(u\) as the \emph{degree} of
  \(F(z)\), denoted \(\deg F(z)\).
\end{definition}

Note that for any \(\alpha \in \FF_{q^m}, \alpha^{[m]} = \alpha^{[0]}\), hence
\(u < m\).  Linearized polynomials can be viewed as linear operators over
\(\FF_{q^m}\), thus their roots form a linear subspace of \(\FF_{q^m}\) with
dimension at most equal to their degrees.  They form an algebra under addition
and the symbolic product, defined as \(L(z) \otimes M(z) \coloneqq L(M(z))\).
Note that the symbolic product is not commutative, so we require both left\=/{}
and right\-/long divisions for linearized polynomials.  These divisions can be
computed by the \emph{extended Euclidean algorithm} for linearized
polynomials~\cite{WSB10}.

Recall that, for an appropriate choice of \(\vec{h}\), the parity\-/check matrix
of a Gabidulin code can be written as
\[
  \mat{H} =
  \begin{bmatrix}
    h_1 & \cdots & h_n \\
    h_1^{[1]} & \cdots & h_n^{[1]} \\
    \vdots & \ddots & \vdots \\
    h_1^{[d - 2]} & \cdots & h_n^{[d - 2]}
  \end{bmatrix}\text.
\]
Suppose we receive a vector \(\vec{y} \coloneqq \vec{c} + \vec{e}\) with
\(\normR{\vec{e}}{q} \eqqcolon w \le \floor{(d - 1) / {2}}\).  The objective is
to determine \(\vec{e}\).  We denote
\(\vec{e} = \begin{bmatrix} E_1 & \cdots & E_w \end{bmatrix}\mat{Y}\), where the
\(E_j \in \FF_{q^m}\) are linearly independent and
\(\mat{Y} \in \FF_q^{w \times n}\) has full rank.  The decoding of a Gabidulin
code defined by the parity\-/check matrix \(\mat{H}\) consists of six steps:
\begin{steps}
\item\label{stp:GC-decoding-1} Calculate the syndrome
  \(\vec{s} = \mat{H} \vec{y}^{\trans} \eqqcolon
  \begin{bmatrix} s_1 & \cdots & s_{d - 1} \end{bmatrix}^{\trans}\).
\item\label{stp:GC-decoding-2} Determine \(\Lambda(z)\) and \(F(z)\) such that
  \(\deg F(z) < w\) and \(F(z) = \Lambda(z) \otimes S(z) \pmod{z^{[d - 1]}}\)
  using the extended Euclidean algorithm.
\item\label{stp:GC-decoding-3} Determine \(w\) linearly independent roots
  \(E_1, \ldots, E_w\) of \(\Lambda(z)\).
\item\label{stp:GC-decoding-4} Determine
  \(\vec{x} \coloneqq \begin{bmatrix} x_1 & \cdots & x_r \end{bmatrix}\) using
  \(\sum_{j = 1}^r E_j x_j^{[p - 1]} = s_p\) for all \(1 \le p \le w\).
\item\label{stp:GC-decoding-5} Determine \(\mat{Y}\) using
  \(x_p = \sum_{j = 1}^n Y_{p, j} h_j\) for all \(1 \le p \le w\).
\item\label{stp:GC-decoding-6} Calculate
  \(\vec{e} = \begin{bmatrix} E_1 & \cdots & E_w \end{bmatrix} \mat{Y}\).
\end{steps}

The complxity of \crefrange{stp:GC-decoding-1}{stp:GC-decoding-6} is
\(\mathcal{O}(d^3 + d n)\).

\section{Description of the GPT Cryptosystem}

In this section, we describe the details of the GPT cryptosystem.  The original
proposal by \textcite{GPT91} was attacked successfully by \textcite{Gib95,
  Gib96}.  A reparation resisting the attack was proposed by \textcite{GO01}.
This reparation was later attacked by \textcite{Ove05, Ove06, Ove08}, which is
the focus of \cref{chap:structural-attacks}.  We also show that more elaborate
reparations can all be reduced to a common form.

\subsection{Original Proposal by \texorpdfstring{\textcite{GPT91}}{Gabidulin,
    Paramonov, and Tretjakov}}

\paragraph{System parameters} Choose \(k, n, m \in \NN\) such that
\(k < n \le m\).  Define the error\-/correcting capacity
\(t \coloneqq \floor{(n - k) / {2}}\).

\paragraph{Key generation} Pick \(\vec{g} \in \FF_{q^m}^n\) with
\(\normR{\vec{g}}{q} = n\), and define \(\mat{G}\) as in
\cref{eq:gabidulin-code-generator}; i.e., as a generator matrix of
\(\Gab_k(\vec{g})\).  Define the \emph{distortion} transformation
\[
  \mathcal{D}\colon \FF_{q^m}^{k \times n} \to \FF_{q^m}^{k \times n}
\]
as
\[
  \mathcal{D}(\mat{G}) \coloneqq \mat{S}(\mat{G} + \mat{X})\text;
\]
where \(\mat{X} \in \FF_{q^m}^{k \times n}\) is a random matrix of prescribed
rank \(t_{\mat{X}}\) and \(\mat{S} \in \GL_k(\FF_{q^m})\).

\paragraph{Public key} Tuple \((\mat{G}_{\pub}, t_{\pub})\), where
\(\mat{G}_{\pub} \coloneqq \mathcal{D}(\mat{G})\) and
\(t_{\pub} \coloneqq t - t_{\mat{X}}\).

\paragraph{Private key} Tuple \((\mat{G}, \mat{S})\).

\paragraph{Encryption} To encrypt a message \(\vec{x} \in \FF_{q^m}^k\), we
choose a random error \(\vec{e} \in \FF_{q^m}^n\) with
\(\normR{\vec{e}}{q} \le t_{\pub}\) and compute the ciphertext
\[
  \vec{y} = \vec{x} \mat{G}_{\pub} + \vec{e}\text.
\]

\paragraph{Decryption} To decrypt a ciphertext \(\vec{y}\), we simply apply the
decoding procedure for Gabidulin codes (\cref{sec:gabidulin-code-decoding}) to
it.  The received word can be decoded since it is corrupted by
\(\vec{x} \mat{S} \mat{X} + \vec{e}\) whose rank is less than or equal to \(t\)
since
\[
  \normR{\vec{x} \mat{S} \mat{X}}{q} \le t_{\mat{X}}
\]
and
\[
  \normR{\vec{x} \mat{S} \mat{X} + \vec{e}}{q} \le
  \normR{\vec{x} \mat{S} \mat{X}}{q} + \normR{\vec{e}}{q} \le
  t\text.
\]

However, \textcites{Gib95, Gib96} showed that the GPT cryptosystem as defined
above is vulnerable to a polynomial\-/time key\-/recovery attack.  Consequently,
\textcite{GO01} proposed a reparation by considering a more general
transformation combining the distortion matrix \(\mat{X}\) with a column
scrambler \(\mat{P}\) over the base field.

\subsection{Reparation by \texorpdfstring{\textcite{GO01}}{Gabidulin and
    Ourivski}}

In order to mask the structure better, a more elaborate distortion
transformation was proposed.  Let \(l \in \NN\) with \(l \ll n\).  The
distortion transformation is now of the form
\begin{equation}\label{eq:distortion-transformation}
  \mathcal{D}(\mat{G}) \coloneqq
  \mat{S}
  \begin{bmatrix} \mat{X}_1 & \mat{G} + \mat{X}_2 \end{bmatrix}
  \mat{P}\text,
\end{equation}
where \(\mat{X}_1 \in \FF_{q^m}^{k \times l}\),
\(\mat{X}_2 \in \FF_{q^m}^{k \times n}\) with \(\rank \mat{X}_2 < t\), and
\(\mat{P} \in \GL_{n + l}(\FF_q)\).  The public generator matrix is again
\(\mat{G}_{\pub} \coloneqq \mathcal{D}(\mat{G})\); while the public rank
parameter is \(t_{\pub} = t - t_{\mat{X}_2}\), where
\(t_{\mat{X}_2} \coloneqq \rank \mat{X}_2\).

The decryption process computes
\[
  \mat{P}^{-1} \coloneqq
  \begin{bmatrix} \mat{Q}_1 & \mat{Q}_2 \end{bmatrix}\text,
\]
where \(\mat{Q}_1 \in \FF_q^{(n + l) \times l}\), and
\(\mat{Q}_2 \in \FF_q^{(l + n) \times n}\).  The last \(n\) components of
\(\vec{y} \mat{P}^{-1}\) form the vector
\[
  \vec{x} \mat{S} \mat{G} + \vec{x} \mat{S} \mat{X}_2 + \vec{e} \mat{Q}_2\text.
\]
Now, since \(\normR{\vec{e} \mat{Q}_2}{q} \le \normR{\vec{e}}{q}\) and
\(\normR{\vec{x} \mat{S} \mat{X}_2}{q} \le \normR{\mat{X}_2}{q}\), it follows
that
\[
  \normR{\vec{x} \mat{S} \mat{X}_2 + \vec{e} \mat{Q}_2}{q} \le t\text.
\]
In other words, applying the decoding procedure for Gabidulin codes to the last
\(n\) components of \(\vec{y} \mat{P}^{-1}\) allows the holder of the private
key to compute \(\vec{x} \mat{S}\), and thus \(\vec{x}\).

\(\mat{X} \coloneqq \begin{bmatrix} \mat{X}_1 & \mat{X}_2 \end{bmatrix} \in \FF_{q^m}^{k \times (l + n)}\)\---
called the \emph{distortion matrix}\--- is needed to mask the structure of
\(\mat{G}\).  Intuitively, if the value of \(t_{\mat{X}_2}\)\--- the
\emph{distortion parameter}\--- is higher, the work factor of any structural
attack increases.  However, a higher value for \(t_{\mat{X}_2}\) implies a
smaller rank of the error, making generic decoding more feasible.
\Textcite{GY05} outlined a procedure for choosing the optimal distortion
parameter.

\subsection{Common Form of the Public Key}

We show that a distortion caused by
\(\mat{X} \coloneqq \begin{bmatrix} \mat{X}_1 & \mat{X}_2 \end{bmatrix}\) with
\(\rank \mat{X}_2 = t_{\mat{X}_2}\) can be concentrated in just
\(t_{\mat{X}_2}\) columns~\cite{Ksh07}.

\begin{theorem}
  Let \(\mat{G}_{\pub}\) be as in \cref{eq:distortion-transformation} and assume
  that \(\rank \mat{X}_2 = t_{\mat{X}_2}\).  Then, there exist
  \(\mat{P}^* \in \GL_{l + n}(\FF_q)\),
  \(\mat{X}^* \in \FF_{q^m}^{k \times (l + t_{\mat{X}_2})}\), and \(\mat{G}^*\)
  which generates an \([n - t_{\mat{X}_2}, k]\)\=/Gabidulin code
  \(\Gab_k(\vec{g}^*)\) such that
  \[
    \mat{G}_{\pub} =
    \mat{S} \begin{bmatrix} \mat{X}^* & \mat{G}^* \end{bmatrix} \mat{P}^*\text.
  \]
  Furthermore, \(\Gab_k(\vec{g}^*)\) can correct more than \(t_{\pub}\) errors.
  \begin{proof}
    Since \(\rank \mat{X}_2 = t_{\mat{X}_2}\), there exist
    \(\mat{T}_2 \in \GL_n(\FF_q)\), and
    \(\mat{X}'_2 \in \FF_{q^m}^{k \times t_{\mat{X}_2}}\) such that
    \[
      \mat{X}_2 \mat{T}_2 = \begin{bmatrix} \mat{X}'_2 & \mat{0} \end{bmatrix}\text.
    \]
    (Consider the process of Gaussian elimination.)  Thus, letting
    \[
      \mat{T} =
      \begin{bmatrix}
        \mat{I}_l & \mat{0} \\
        \mat{0} & \mat{T}_2
      \end{bmatrix}\text,
    \]
    we obtain
    \[
      \begin{aligned}
        \mat{G}_{\pub} &= \mat{S} \begin{bmatrix} \mat{X}_1 & \mat{G} + \mat{X}_2 \end{bmatrix} \mat{P} \\
        &= \mat{S} \begin{bmatrix} \mat{X}_1 & \mat{G} \mat{T}_2 + \mat{X}_2 \mat{T}_2 \end{bmatrix} \mat{T}^{-1} \mat{P} \\
        &\eqqcolon \mat{S} \begin{bmatrix} \mat{X}_1 & \mat{G}' + \mat{X}_2 \mat{T}_2 \end{bmatrix} \mat{Q}\text;
      \end{aligned}
    \]
    where \(\mat{G}' \coloneqq \mat{G} \mat{T}_2\), and
    \(\mat{Q} \coloneqq \mat{T}^{-1} \mat{P}\).  \(\mat{G}'\) generates the
    \([n, k]\)\=/Gabidulin code \(\Gab_k(\vec{g}')\), with
    \(\vec{g}' = \vec{g} \mat{T}_2\).  If we decompose \(\mat{G}'\) as
    \(\begin{bmatrix} \mat{G}'_1 & \mat{G}'_2 \end{bmatrix}\), where
    \(\mat{G}'_1 \in \FF_{q^m}^{k \times t_{\mat{X}_2}}\), and
    \(\mat{G}'_2 \in \FF_{q^m}^{k \times (n - t_{\mat{X}_2})}\), we obtain
    \[
      \mat{G}' + \mat{X}_2 \mat{T}_2 = \begin{bmatrix} \mat{G}'_1 + \mat{X}'_2 & \mat{G}'_2 \end{bmatrix}\text.
    \]
    By setting
    \(\mat{X}^* = \begin{bmatrix} \mat{X}_1 & \mat{G}'_1 + \mat{X}'_2 \end{bmatrix}\),
    we get the desired result.  \(\mat{G}'_2\) generates the
    \([n - t_{\mat{X}_2}, k]\)\=/Gabidulin code \(\Gab_k(\vec{g}'_2)\), where
    \(\vec{g}'_2 = \begin{bmatrix} g'_{t_{\mat{X}_2} + 1} & \cdots & g'_n \end{bmatrix}\).
    The error\-/correction capacity \(t^*\) of \(\Gab_k(\vec{g}'_2)\) is
    \((n - t_{\mat{X}_2} - k) / {2} = t - t_{\mat{X}_2} / {2} \implies t^* > t - t_{\mat{X}_2}\).
  \end{proof}
\end{theorem}

\chapter{Structural Attacks Against GPT}\label{chap:structural-attacks}

Gabidulin codes possess a rich structure that has been exploited to mount
attacks on the GPT cryptosystem.  In this chapter, we review the most notable of
these attacks: Overbeck's attack.  We prove that Overbeck's attack applies to a
very general form of the GPT cryptosystem, and then state the properties that
make the attack possible.  At the end, we formulate Loidreau's recent reparation
which avoids Overbeck's attack.

\section{Intuition Behind the Attacks}

Let \(\mathcal{C}\) be an \([n, k, d]\)\=/Gabidulin code generated by
\(\mat{G}\).  Recall that \(\mat{G}\) is of the form
\[
  \begin{bmatrix}
    g_1 & \cdots & g_n \\
    g_1^{[1]} & \cdots & g_n^{[1]} \\
    \vdots & \ddots & \vdots \\
    g_1^{[k - 1]} & \cdots & g_n^{[k - 1]}
  \end{bmatrix}\text.
\]
Now, one can see that
\[
  \mathcal{C}^{[1]} \cup \mathcal{C}
\]
represents a Gabidulin code of dimension \(k - 1\).  (\(\mathcal{C}^{[1]}\)
denotes the code generated by \(\mat{G}^{[1]}\).)  This was the basis of a
polynomial time algorithm to retrieve the hidden Gabidulin structure in the GPT
cryptosystem.

\section{Distinguishing Properties}

We now formally state the distinguishing properties of Gabidulin codes that have
been exploited by \textcite{Ove05, Ove06, Ove08} to mount successful attacks.

\begin{definition}
  For any \(i \in \NN\), let
  \(\Lambda_i\colon \FF_{q^m}^{k \times n} \to \FF_{q^m}^{i k \times n}\) be
  the \(\FF_q\)\=/linear operator defined as:
  \[
    \Lambda_i(\mat{X}) \coloneqq
    \begin{bmatrix}
      \mat{X}^{[0]} \\
      \mat{X}^{[1]} \\
      \vdots \\
      \mat{X}^{[i]}
    \end{bmatrix}\text.
  \]
\end{definition}

For any code \(\mathcal{C}\) generated by \(\mat{G}\), we denote by
\(\Lambda_i(\mathcal{C})\) the code generated by \(\Lambda_i(\mat{G})\).

\begin{lemma}
  Let \(\vec{g} \in \FF_{q^m}^n\) with \(\normR{\vec{g}}{q} = n\).  For
  \(k, i \in \NN\) such that \(k \le n\) and \(i \le n - k - 1\), we have:
  \[
    \Lambda_i(\Gab_k(\vec{g})) = \Gab_{k + i}(\vec{g})\text.
  \]
\end{lemma}

On the other hand, if \(\mat{G}\) is a randomly\-/drawn matrix, we obtain
something quite different.

\begin{lemma}
  If \(\mathcal{C} \subset \FF_{q^m}^n\) is a code generated by a random matrix
  \(\mat{G} \in \FF_{q^m}^{k \times n}\),
  \[
    \dim \Lambda_i(\mathcal{C}) = \min\{n, (i + 1) k\}
  \]
  with high probability.
\end{lemma}

\section{Overbeck's Attack}

\begin{lemma}
  Let
  \(\mat{P} = \begin{bsmallmatrix} \mat{A} & \mat{0} \\ \mat{C} & \mat{D} \end{bsmallmatrix}\)
  where \(\mat{A}\) and \(\mat{D}\) are square matrices.  \(\mat{P}\) is
  nonsingular if and only if \(\mat{A}\) and \(\mat{D}\) are nonsingular, and
  \[
    \mat{P}^{-1} =
    \begin{bmatrix}
      \mat{A}^{-1} & \mat{0} \\
      -\mat{D}^{-1} \mat{C} \mat{A}^{-1} & \mat{D}^{-1}
    \end{bmatrix}\text.
  \]
\end{lemma}

Assume that
\(\mat{G}_{\pub} = \mat{S} \begin{bmatrix} \mat{X} & \mat{G}\end{bmatrix} \mat{P}\)
is the public generator matrix with \(\mat{P} \in \GL_{l + n}( \FF_q)\),
\(\mat{X} \in \FF_{q^m}^{k \times l}\) and \(\mat{G}\) generates a Gabidulin
code \(\Gab_k(\vec{g})\).  Observe that
\[
  \Lambda_i(\mat{G}_{\pub}) =
  \mat{S}_{\ext} \begin{bmatrix} \Lambda_i(\mat{X}) & \Lambda_i(\mat{G}) \end{bmatrix} \mat{P}\text,
\]
where
\[
  \mat{S}_{\ext} \coloneqq
  \begin{bmatrix}
    \mat{S}^{[0]} & \cdots & \mat{0} \\
    \vdots & \ddots & \vdots \\
    \mat{0} & \cdots & \mat{S}^{[i]}
  \end{bmatrix}\text.
\]
Since \(\Lambda_i(\mat{G})\) generates \(\Gab_{k + i}(\vec{g})\), there exists
\(\mat{S}' \in \GL_{k (i + 1)}(\FF_{q^m})\) such that
\begin{equation}\label{eq:extended-public-key}
  \mat{S}' \Lambda_i(\mat{G}_{\pub}) =
  \begin{bmatrix}
    \mat{X}^* & \mat{G}_{k + i} \\
    \mat{X}^{**} & \mat{0}
  \end{bmatrix}\text.
\end{equation}
Using \cref{eq:extended-public-key}, one can see that when \(i = n - k - 1\),
\[
  \dim \Lambda_{n - k - 1}(\mathcal{C}_{\pub}) = n - 1 + \rank \mat{X}^{**}\text.
\]
When \(\rank \mat{X}^{**} = l\),
\(\dim \Lambda_{n - k - 1}(\mathcal{C}_{\pub}) = n + l - 1\), and thus
\(\dim {\Lambda_{n - k - 1}(\mathcal{C}_{\pub})}^{\perp} = 1\).  If
\(\vec{h} \in {\Gab_{n - 1}(\vec{g})}^{\perp}\) is a nonzero vector, and we set
\(\vec{h}^* = \begin{bmatrix} \vec{0} & \vec{h} \end{bmatrix} {(\mat{P}^{-1})}^{\trans}\),
then\--- under the assumption that \(\rank \mat{X}^{**} = l\)\--- we have
\begin{equation}\label{eq:spanning-the-dual}
  {\Lambda_{n - k - 1}(\mathcal{C}_{\pub})}^{\perp} = \lspan\{\vec{h}^*\}.
\end{equation}

\begin{theorem}
  Let \(\vec{v} \in {\Lambda_{n - k - 1}(\mathcal{C}_{\pub})}^{\perp}\) be a
  nonzero vector.  Any matrix \(\mat{T} \in \GL_{l + n}(\FF_q)\) that satisfies
  \(\vec{v} \mat{T} = \begin{bmatrix} \vec{0} & \vec{h}' \end{bmatrix}\) with
  \(\vec{h}' \in \FF_{q^m}^n\) is an alternative column scrambler; i.e., there
  exist \(\mat{Z} \in \FF_{q^m}^{k \times l}\) and \(\mat{G}^*\) generating
  \(\Gab_k(\vec{g}^*)\) such that
  \[
    \mat{G}_{\pub} = \mat{S} \begin{bmatrix} \mat{Z} & \mat{G}^* \end{bmatrix} \mat{T}\text.
  \]
  \begin{proof}
    From \cref{eq:spanning-the-dual}, there exists \(\alpha \in \FF_{q^m}\) such
    that
    \(\vec{v} = \alpha \vec{h}^* = \begin{bmatrix} \vec{0} & \alpha \vec{h} \end{bmatrix} {(\mat{P}^{-1})}^{\trans}\),
    where \(\vec{h} \in {\Gab_{n - 1}(\vec{g})}^{\perp}\) is a nonzero vector.
    Let \(\mat{T} \in \GL_{l + n}(\FF_q)\) be such that
    \(\vec{v} \mat{T}^{\trans} = \begin{bmatrix} \vec{0} & \vec{h}' \end{bmatrix}\),
    and consider the matrices \(\mat{A} \in \FF_q^{l \times l}\) and
    \(\mat{D} \in \FF_q^{n \times n}\) so that
    \[
      \mat{T} \mat{P}^{-1} =
      \begin{bmatrix}
        \mat{A} & \mat{B} \\
        \mat{C} & \mat{D}
      \end{bmatrix}\text.
    \]
    Now, we have
    \[
      \vec{v} \mat{T}^{\trans} =
      \begin{bmatrix} \vec{0} & \alpha \vec{h} \end{bmatrix}
      {(\mat{P}^{-1})}^{\trans} \mat{T}^{\trans} =
      \begin{bmatrix} \vec{0} & \alpha \vec{h} \end{bmatrix}
      {(\mat{P}^{-1} \mat{T})}^{\trans} =
      \begin{bmatrix} \vec{0} & \vec{h}' \end{bmatrix}\text.
    \]
    Notice that
    \(\vec{h} \mat{B}^{\trans} = \vec{0} \implies \mat{B} = \mat{0}\) since
    \(\normR{\vec{h}}{q} = n\).  Thus,
    \[
      \mat{T} \mat{P}^{-1} =
      \begin{bmatrix} \mat{A} & \mat{0} \\ \mat{C} & \mat{D} \end{bmatrix}
      \implies
      \mat{P} \mat{T}^{-1} =
      \begin{bmatrix} \mat{A}' & \mat{0} \\ \mat{C}' & \mat{D}' \end{bmatrix}\text.
    \]
    From here,
    \[
      \mat{G}_{\pub} \mat{T}^{-1} =
      \mat{S}
      \begin{bmatrix} \mat{X} & \mat{G} \end{bmatrix}
      \begin{bmatrix} \mat{A}' & \mat{0} \\ \mat{C}' & \mat{D}' \end{bmatrix} =
      \mat{S}
      \begin{bmatrix} \mat{Z} & \mat{G}^* \end{bmatrix}\text,
    \]
    where \(\mat{G}^* \coloneqq \mat{G} \mat{D}'\) is a generator matrix of an
    \([n, k]\)\=/Gabidulin code.
  \end{proof}
\end{theorem}

\section{Defending Against Overbeck's Attack}

As we saw, Overbeck's attack is quite general as it can be applied to a wide
variety of GPT\=/like cryptosystems.  In order to defend against it, we need to
see what makes it feasible to perform the attack.  There are two key properties:
\begin{properties}
\item\label{prop:column-scrambler} The column scrambler \(\mat{P}\) is defined
  over the base field; and
\item\label{prop:codimension} The codimension of
  \(\Lambda_{n - k - 1}(\mathcal{C})\) is \(1\).
\end{properties}
Here, we focus on \cref{prop:column-scrambler}.  We can relax the optimality of
the code by scrambling the columns with a nonisometry of the metric.  This was
already done in the Hamming metric in the case of GRS codes.

\begin{lemma}\label{lem:rank-multiplication}
  Let \(\mathcal{V} = \{\beta_1, \ldots, \beta_{\lambda}\}\) be the
  \(\FF_q\)\=/linear subspace generated by
  \(\{\beta_1, \ldots, \beta_{\lambda}\}\).  Let
  \(\mat{P} \in \GL_n(\mathcal{V})\).  Then,
  \[
    \forall \vec{x} \in \FF_{q^m}^n\ldotp
    \normR{\vec{x} \mat{P}}{q} \le \lambda \normR{\vec{x}}{q}\text.
  \]
\end{lemma}

\begin{corollary}
  Let \(\mathcal{C}\) be an \([n, k, d]\)\=/Gabidulin code over \(\FF_{q^m}\).
  Let \(\mathcal{V}\) be a \(\lambda\)\=/dimensional subspace of \(\FF_{q^m}\),
  and let \(\mat{P} \in \GL_n(\mathcal{V})\).  Then,
  \[
    \mathcal{C} \mat{P}^{-1} \coloneqq \{\vec{c} \mat{P}^{-1} : \vec{c} \in \mathcal{C}\}
  \]
  has dimension \(k\) and minimum distance \(d' \ge \floor{c / {\lambda}}\).
  \begin{proof}
    Since \(\mat{P}\) is nonsingular, \(\mathcal{C}\) and
    \(\mathcal{C} \mat{P}^{-1}\) have the same dimension.  Now, suppose
    \(d' < d / {\lambda}\), and let \(\vec{c} \in \mathcal{C} \mat{P}^{-1}\) be
    a nonzero vector with \(\normR{\vec{c}}{q} = d'\).  By construction,
    \(\vec{c} \vec{P} \in \mathcal{C}\), and by \cref{lem:rank-multiplication},
    \(\normR{\vec{c} \mat{P}}{q} \le d' \lambda < d \implies \vec{c} \mat{P} = \vec{0}\).
    Thus, \(\vec{c} = \vec{0}\), which contradicts the hypothesis.
  \end{proof}
\end{corollary}

We can use this property to design a cryptosystem where the column scrambler
\(\mat{P}\) is defined over the extension field.

\section{Reparation by \texorpdfstring{\textcite{Loi17}}{Loidreau}}

\Textcite{Loi17} proposed and analyzed the security of a cryptosystem based on
the aforementioned ideas.  What follows is a description of the system and the
parameters proposed by the author.

\paragraph{Private key}
\begin{enumerate}
\item Gabidulin code \(\Gab_k(\vec{g})\) with \(\normR{\vec{g}}{q} = n\),
  generator matrix \(\mat{G}\) as in \cref{eq:gabidulin-code-generator}, and
  error\-/correction capacity \(t \coloneqq \floor{(n - k) / {2}}\).
\item \(\mat{S} \in \GL_k(\FF_{q^m})\), the \emph{row scrambler}.
\item \(\mathcal{V}\), a \(\lambda\)\=/dimensional subspace of \(\FF_{q^m}\);
  and \(\mat{P} \in \GL_n(\mathcal{V})\), the \emph{column scrambler}.
\end{enumerate}

\paragraph{Public key} Tuple \((\mat{G}_{\pub}, t_{\pub})\), where
\[
  \mat{G}_{\pub} \coloneqq
  \mat{S} \mat{G} \mat{P}^{-1} \ \text{and}\ t_{\pub} =
  \ceil*{\frac{n - k}{2 \lambda}}\text.
\]

\paragraph{Encryption} To encrypt a message \(\vec{x} \in \FF_{q^m}^k\), we
choose a random error \(\vec{e} \in \FF_{q^m}^n\) with
\(\normR{e}{q} \le t_{\pub}\) and compute the ciphertext:
\(\vec{y} = \vec{x} \mat{G}_{\pub} + \vec{e}\).

\paragraph{Decryption} Compute
\(\vec{y} \mat{P} = \vec{x} \mat{S} \mat{G} + \vec{e} \mat{P}\).  We know that
\[
  \normR{\vec{e} \mat{P}}{q} \le \lambda \ceil*{\frac{n - k}{2 \lambda}} \le \ceil*{\frac{n - k}{2}}\text,
\]
thus, we can decode using a decoding algorithm for \(\Gab_k(\vec{g})\)
to obtain \(\vec{x} \mat{S}\).  From here, we can retreive \(\vec{x}\)
immediately.

\section{Analysis of the Cryptosystem}

Having the column scrambler defined over the extension field nullifies
Overbeck's attack since it is no longer true that
\[
  \mat{P}^{[i]} = \mat{P}\text.
\]
As a matter of fact, there is no known reason for the public key to have any
particular structure.  Namely, even though the entires of \(\mat{P}\) are all in
some \(\lambda\)\=/dimensional subspace \(\mathcal{V}\), there is no reason for
the entries of \(\mat{P}^{-1}\) to be in any subspace.  The proposed parameters
can be found in \cref{tab:loidreau-parameters}.

\begin{table}
  \centering
  \captionabove{Proposed parameters for the Loidreau
    cryptosystem}\label{tab:loidreau-parameters}
  \begin{tabular}{S[table-format=3]S[table-format=3]S[table-format=2]S[table-format=1]S[table-format=3]S[table-format=2.1]}
    \toprule
    {\(m\)} & {\(n\)} & {\(k\)} & {\(\lambda\)} & {Claimed security (\si{bit})} & {Key size (\si{\kilo\byte})} \\
    \midrule
    50 & 50 & 32 & 3 & 64 & 3.6 \\
    80 & 80 & 11 & 3 & 110 & 8.3 \\
    96 & 64 & 40 & 3 & 120 & 11.5 \\
    128 & 90 & 24 & 3 & 240 & 21.5 \\
    128 & 120 & 80 & 5 & 240 & 51 \\
    \bottomrule
  \end{tabular}
\end{table}

\chapter{Information Set Decoding}\label{chap:ISD}

As we saw, decoding random linear codes is a fundamental problem not only in
coding theory, but also in cryptography.  We established that it is a
\enquote{very hard} problem and is conjectured to be hard on average.  We saw a
family of codes\--- Gabidulin codes\--- which can be decoded efficiently, but
most codes are not known to have an efficient decoding algorithm.

There are several known techniques for decoding random linear codes.  In the
Hamming metric, the best algorithms that we know of rely on information set
decoding (ISD), originally proposed by \textcite{Eli78} and inspired by the work
of \textcite{Pra62}.  In short, the idea behind ISD is to pick a sufficiently
large set of error\-/free positions in a received word.  Then, the information
sequence can be obtianed by linear algebra.

Since its introduction, ISD has been refined to the point of being able to
decode random binary linear codes in the Hamming metric with complexity
\(\mathcal{O}(2^{n / {20}})\)~\cite{B+12}.  However, ISD has not been explored
in the context of the rank metric.  In this chapter, we introduce and analyze
ISD algorithms suitable for the rank metric.  First, we introduce the MinRank
problem in \cref{sec:MR-definition,sec:MR-attacks}, as it will be used as a
subproblem is one of the ISD algorithms.

\section{The MinRank Problem}\label{sec:MR-definition}

\begin{definition}[MinRank over a field]\label{def:MR}
  Let \(\KK\) be a field.  Let \(\mat{M}_0; \mat{M}_1, \ldots, \mat{M}_m\) be
  matrices from \(\KK^{\mu \times \nu}\), and let \(r \in \NN\).  The MinRank
  problem instance asks us to find a vector
  \(\vec{\alpha} \coloneqq \begin{bmatrix} \alpha_1 & \cdots &
    \alpha_m \end{bmatrix} \in \KK^m\) such that
  \[
    \rank\left(\sum_{i = 1}^{m} \alpha_i \mat{M}_i - \mat{M}_0\right) \le r\text.
  \]
  We denote by
  \(\MR(m, \mu, \nu, r, \KK; \mat{M}_0; \mat{M}_1, \ldots, \mat{M}_m)\) the set
  of all solutions to the problem.
  \begin{remark}
    Typically, \(\KK \coloneqq \FF_q\) for \(q\) (power of a) prime.  We will
    implicitly assume this hereafter.
  \end{remark}
\end{definition}

MinRank is known to be \NP\=/complete.  As a matter of fact, any system of
multivariate polynomial equations can be encoded as a MinRank instance.  More
interestingly, \textsc{Minimum Distance Decoding} in the rank metric can be
reduced to solving a MinRank instance.

\begin{theorem}\label{thm:RD-to-MR}
  Let \(\mathcal{C}\) be an \([n, k]\)\=/code over \(\FF_{q^m}\) in the rank
  metric.  There is a many\-/to\-/one reduction from
  \(\MDD(\mathcal{C}, \vec{y}, w)\) to
  \(\MR(m k, m, k, w, \FF_q; \mat{M}_0; \mat{M}_1, \ldots, \mat{M}_{m k})\).
  \begin{proof}[Proof sketch]
    Each entry of \(\vec{y} - \vec{c} \eqqcolon \vec{e}\) can be written in a
    basis \(\mathcal{B} = (\vec{\beta}_1, \ldots, \vec{\beta}_m)\).  For each of
    the \(k\) coordinates of \(\vec{c} = \vec{x} \mat{G}\), we can
    \enquote{expand} each \(\vec{\beta}_i\) into an \(m \times k\) matrix over
    \(\FF_q\), for a total of \(m k\) such matrices.
  \end{proof}
\end{theorem}

\section{Solving MinRank Instances}\label{sec:MR-attacks}

The two most practical and most studied methods of solving MinRank instances
are:
\begin{itemize}
\item Guessing the kernel of the solution\--- the so\-/called \enquote{kernel
    attack}; and
\item Modeling a MinRank instance as a system of polynomial (multivariate
  quadratic) equations.
\end{itemize}

\subsection{The Kernel Attack}\label{sec:kernel-attack}

This was the first non-trivial attack against MinRank and was proposed by
\textcite{GC00}.  The main idea is that instead of guessing \(\vec{\alpha}\), we
can guess the kernel of the resulting matrix.  Then, assuming that the kernel
was guessed correctly, we can \emph{solve} for \(\vec{\alpha}\).

For some parameter \(\vec{\beta} \in \FF_q^m\), denote by \(H(\vec{\beta})\) the
linear combination
\[
  \sum_{i=1}^m \beta_i \mat{M}_i - \mat{M}_0\text.
\]
We would like to have \(\rank H(\vec{\beta}) \le r\), i.e., \(\vec{\beta}\) to
be a solution.  If that were the case, then by the rank\-/nullity theorem of
linear algebra, we would have \(\dim \ker H(\vec{\beta}) \ge \nu - r\).

We can proceed by randomly choosing $\kappa$ vectors
\(\vec{x}^{(i)} \in \FF_q^{\nu}\) for \(\ 1 \le i \le \kappa\).  If these
vectors are such that they all fall into the kernel of a solution to the MinRank
instance, then solving for \(\vec{\beta}\) in
\begin{equation}\label{eq:kernel-attack}
  \setlength{\arraycolsep}{0pt}
  \newcolumntype{B}{>{{}}c<{{}}}
  \left\{
    \begin{array}{l B l}
      H(\vec{\beta}) \vec{x}^{(1)} & = & \vec{0} \\
      H(\vec{\beta}) \vec{x}^{(2)} & = & \vec{0} \\
                                   & \vdots & \\
      H(\vec{\beta}) \vec{x}^{(\kappa)} & = & \vec{0}
    \end{array}
  \right.
\end{equation}
would allow us to retrieve the correct solution.  \Cref{eq:kernel-attack} is a
system in \(\kappa \mu\) equations and \(m\) unknowns.  Since we would like
\cref{eq:kernel-attack} to never be under\-/determined and have its numbers of
equations and unknowns match as closely as possible, we can choose
\(\kappa \coloneqq \ceil{m / {\mu}}\).

It was established that there are at least \(q^{\nu - r}\) vectors in
\(\ker H(\vec{\beta})\) whenever \(\vec{\beta}\) is a solution.  Having that in
mind, we can see that
\[
  \Pr(\{\vec{x}^{(1)}, \ldots, \vec{x}^{(\kappa)}\} \subseteq \ker H(\vec{\beta})) \ge q^{-\kappa r} = q^{-\ceil{m / {\mu}} r}\text.
\]
From here, we get an overall complexity of
\(\mathcal{O}(m {(\ceil{m / {\mu}} \mu)}^2 q^{\ceil{m / {\mu}} r})\).  The
takeaway is that the kernel attack is significantly better than a brute\-/force
enumeration whenever \(\ceil{m / {\mu}} r \ll m\), which is almost always the
case.

\subsection{As a System of Multivariate Quadratic Equations}

Notice that instead of guessing, we can explicitly construct the kernel's
solution, i.e., treat the \(\vec{x}^{(i)}\) in \cref{eq:kernel-attack} as
unknowns.  This modeling was first proposed by \textcite{KS99}.

Define \(H(\vec{\beta})\) as in \cref{sec:kernel-attack}.  We already saw that
whenever \(\vec{\beta}\) is a solution, there are at least \(\nu - r\) linearly
independent vectors in \(\ker H(\vec{\beta})\).  Proceed by trying to fix these
vectors as follows:
\begin{equation}\label{eq:MR-fixed-vectors}
  \setlength{\savedcolsep}{\arraycolsep}
  \setlength{\arraycolsep}{0pt}
  \newcolumntype{B}{>{{}}c<{{}}}
  \begin{array}{l B l}
    \vec{x}^{(1)} & \coloneqq
    & \setlength{\arraycolsep}{\savedcolsep}%
      \begin{bmatrix} 1 & 0 & \cdots & 0 & x^{(1)}_1 & \cdots & x^{(1)}_r \end{bmatrix}^{\trans}
    \\
                  & \vdots &
    \\
    \vec{x}^{(\nu - r)}
                  & \coloneqq
    & \setlength{\arraycolsep}{\savedcolsep}%
      \begin{bmatrix} 0 & 0 & \cdots & 1 & x^{(\nu - r)}_1 & \cdots & x^{(\nu - r)}_r \end{bmatrix}^{\trans}\text.
  \end{array}
\end{equation}
\begin{remark}
  \Cref{eq:MR-fixed-vectors} is true up to a \emph{change of basis}, so it might
  happen that it does not arrive at a solution.
\end{remark}

Bearing this in mind, one can see that the multivariate quadratic (MQ) system
\begin{equation}\label{eq:MR-as-MQ}
  \left(\sum_{i = 1}^m \beta_i \mat{M}_i - \mat{M}_0\right)
  \begin{bmatrix}
    1 & 0 & \cdots & 0 \\
    0 & 1 & \cdots & 0 \\
    \vdots & \vdots & \ddots & \vdots \\
    0 & 0 & \cdots & 1 \\
    x^{(1)}_1 & x^{(2)}_1 & \cdots & x^{(n - r)}_1 \\
    \vdots & \vdots & \ddots & \vdots \\
    x^{(1)}_r & x^{(2)}_r & \cdots & x^{(n - r)}_r
  \end{bmatrix} = \mat{0}_{\mu \times (\nu - r)}
\end{equation}
over \(\FF_q[\beta_1, \ldots, \beta_m, x^{(1)}_1, \ldots, x^{(\nu - r)}_r]\)
consists of \(\mu (\nu - r)\) equations and \(r (\nu - r) + \mu\) unknowns.  The
complexity of solving \cref{eq:MR-as-MQ} is hard to evaluate and depends a lot
on the method.  Nevertheless, it is exponential in the worst case.

\Textcite{KS99} proposed solving \cref{eq:MR-as-MQ} by \emph{relinearization}.
However, Gröbner bases are a more popular approach for solving any system of
polynomial equations.  One can think of them as analogous to the process of
Gaussian elimination.

\paragraph{Solving \cref{eq:MR-as-MQ} by Gröbner bases} \Textcite{FLP08}
proposed a one\-/to\-/one correspondence between the affine variety
\[
  \mathcal{V}(\mathcal{I}_{\KS}) \coloneqq
  \{\vec{x} \in \FF_q^{r (\nu - r) + m} : \forall f \in \mathcal{I}_{\KS}\ldotp f(\vec{x}) = 0\}
\]
and the solution to \cref{eq:MR-as-MQ}, where \(\mathcal{I}_{\KS}\) is the ideal
generated by the equations of \cref{eq:MR-as-MQ}.  This allows one to use
Gröbner bases to solve the system.  The theoretical complexity of computing a
Gröbner basis of a polynomial system with \(m\) equations and \(n\) unknowns is
given by
\[
  \mathcal{O}\left(m \binom{n + d_{\reg}}{d_{\reg}}^{\omega}\right)\text,
\]
where \(d_{\reg}\) is the degree of regularity (i.e., the highest degree reached
during the computation), and \(2 \le \omega \le 3\) is the exponent in the
complexity of matrix multiplication.  However, in practice\--- and especially
for MinRank instances since the equations of \cref{eq:MR-as-MQ} are bilinear\---
the complexity is much lower~\cite{FLP08} and can often be experimentally
bounded by a polynomial.

\section{The Information Set Decoding Algorithms}

In this section, we formulate and analyze two variants of information set
decoding algorithms suitable for decoding in the rank metric:
\begin{enumerate*}[label={(\arabic*)}]
\item algorithms using sets of coordinates (\cref{sec:ISD-R-SC}); and
\item algorithms using projections onto subspaces (\cref{sec:ISD-R-P}).
\end{enumerate*}
The former are a straightforward adaptation of the algorithms in the Hamming
metric, while the latter use projections as the \enquote{rank metric analog}
for sets of coordinates.

\subsection{Algorithms in the Hamming Metric}\label{sec:ISD-H}

All information set decoding algorithms studied in this thesis can be described
in two steps:
\begin{steps}
\item\label{stp:ISD-1} Choosing a random information set and assuming it
  satisfies certain condition w.r.t.\ the error pattern; and
\item\label{stp:ISD-2} Decoding the received word \(\vec{y}\) assuming that the
  information set chosen in \cref{stp:ISD-1} is as assumed.
\end{steps}

In this section, we are going to restrict our attention to \emph{binary} codes
in the Hamming metric.  The definition of an information set
(\cref{def:information-set}) immediately lends itself to the simplest ISD
algorithm, \emph{plain information set decoding} (\PISD).

\emph{Plain information set decoding} randomly chooses an information set and
then computes the information by solving linear equations.  In other words, it
assumes that there are \emph{no errors} within the \(k\) positions indexed by
the information set.  Let \(\vec{x} \in \FF_2^k\) be the unknown message, and
\[
  \vec{y} \coloneqq \vec{x} \mat{G} + \vec{e} \in \FF_2^n
\]
be the received word.  Suppose we pick an information set
\(\mathcal{I} \subset \{1, \ldots, n\}\) with \(\card{\mathcal{I}} = k\).  If
the \(\mathcal{I}\)\=/indexed coordinates of \(\vec{y}\) are error\-/free,
i.e., if \(\vec{e}_{\mathcal{I}} = \vec{0}\), then we can compute \(\vec{x}\)
by simple linear algebra:
\[
  \vec{y}_{\mathcal{I}} \mat{G}_{\mathcal{I}}^{-1} =
  ({(\vec{x} \mat{G})}_{\mathcal{I}} + \vec{e}_{\mathcal{I}}) \mat{G}_{\mathcal{I}}^{-1} =
  \vec{x}\text.
\]
This idea originated from \textcite{Pra62}.  In order to analyze the complexity
of the algorithm, we need to determine the probability of success, i.e., picking
an error\-/free information set, and the complexity of computing \(\vec{x}\).


Denote by \(\supp(\vec{e})\) the set of all coordinates of the nonzero entries
of \(\vec{e}\), \(\supp(\vec{e}) = \{i : e_i \ne 0\}\).  Then, the probability
of success is
\[
  \Pr(\mathcal{I} \cap \supp(\vec{e}) = \emptyset) = \binom{n - k}{w}\binom{n}{w}^{-1}\text,
\]
where \(w\) is the Hamming weight of \(\vec{e}\).  To see that this is the case,
notice that there are \(\binom{n - k}{w}\) possible choices for \(\vec{e}\) if
the information set is to be error\-/free; whereas there are \(\binom{n}{w}\)
choices if this restriction is lifted.  The cost of decoding \(\vec{y}\) is
simply the cost of computing the inverse of a matrix by Gaussian elimination.
It is common to consider it as \(\frac{(n - k) k^2}{2}\), although significant
improvements are known.

The algorithm performs a series of independent iterations, where each iteration
consists of a randomly chosen information set \(\mathcal{I}\).  As justified by
\cref{thm:las-vegas-running-time}, the expected running time is
\[
  \binom{n - k}{w}^{-1}\binom{n}{w}\text.
\]

\begin{theorem}\label{thm:las-vegas-running-time}
  Consider a Las Vegas algorithm which succeeds to produce a correct answer with
  probability \(\rho\) and outputs \(\bot\) with probability \(1 - \rho\).  Let
  \(X\) be a random variable that represents the running time of the algorithm.
  Then, its expected running time is
  \[
    \E[X] = \frac{1}{p}\text.
  \]
  \begin{proof}
    Since we will run the algorithm until we observe a correct answer, \(X\)
    follows a geometric distribution.  More specifically, the probability that
    we have to run the algorithm \(i\) times is given by
    \[
      \Pr(X = i) = \rho {(1 - \rho)}^{i - 1}\text.
    \]
    It is a well\-/known result that a random variable with such a distribution
    has expectation equal to \(1 / {\rho}\).
  \end{proof}
\end{theorem}

\Textcite{LB88} explored a generalization of plain information set decoding,
allowing a set of \(p \in \{0, \ldots, w\}\) errors in the information set.
The parameter \(p\) is typically chosen to be a small number; \textcite{Pet11}
proved that \(p = 2\) is in fact optimal in the binary case, improving the
complexity by a factor \(\Theta(n \log n)\) over plain information set decoding
(\(p = 0\)).  Assuming that \(\vec{e}\) is uniformly random, the probability of
having exactly \(p\) errors in the information set is
\[
  \Pr(\card{\mathcal{I} \cap \supp(\vec{e})} = p) =
  \binom{n - k}{w - p}\binom{k}{p}\binom{n}{w}^{-1}\text,
\]
which also gives the probability of success in one iteration of the
Lee\--Brickell algorithm.

\subsection{Algorithms Using Sets of Coordinates in the Rank
  Metric}\label{sec:ISD-R-SC}

The algorithms outlined in \cref{sec:ISD-H} employ the \enquote{classical}
definition of an information set (\cref{def:information-set}).  As we already
mentioned, they have been the most successful way of attacking McEliece\-/like
cryptosystems, but have not been explored in the context of the rank metric.  We
formulate both plain information set decoding and the improvement by
\textcite{LB88}, and analyze their applicability in the rank metric.  We also
provide some thoughts on the feasibility applying Stern's and Dumer's
ideas~\cite{Ste89, Dum96}.

\subsubsection{Plain Information Set Decoding}

We saw that \PISD{} randomly chooses an information set and assumes that the
error indexed by it is of weight \(0\).  This is a very natural notion for the
Hamming metric; however, for the rank metric, it translates to the submatrix
formed by the \(\mathcal{I}\)\=/indexed coordinates being the zero matrix.
\Cref{alg:P-ISD-R-SC} summarizes the procedure.

\begin{algorithm}
  \caption{\PISD{} using sets of coordinates in the rank
    metric}\label{alg:P-ISD-R-SC}
  \DontPrintSemicolon{}%
  \KwIn{Generator matrix \(\mat{G} \in \FF_{q^m}^{k \times n}\), received word
    \(\vec{y} \in \FF_q^n\), and target weight \(w \in \NN\)}%
  \KwOut{Error \(\vec{e} \in \FF_q^n\) with \(\normR{\vec{e}}{q} \le w\) and
    \(\vec{y} - \vec{e} \in \mathcal{C}\)}%
  \Begin{%
    \Repeat{a solution is found}{%
      \(\mathcal{I} \sample \{1, \ldots, n\}\) with
      \(\card{\mathcal{I}} = k\)\;%
      \(\vec{e}' \gets
      \vec{y} - \vec{y}_{\mathcal{I}} \mat{G}_{\mathcal{I}}^{-1} \mat{G}\)\;%
      \lIf{\(\normR{\vec{e}'}{q} \le w\)}{\Return{\(\vec{e}'\)}}%
    }%
  }
\end{algorithm}

Subsequent experiments will show that this algorithm is not applicable to the
rank metric.  Indeed, there is just a single matrix of rank \(0\) (the zero
matrix), and the probability of the \(\mathcal{I}\)\=/indexed coordinates of
\(\vec{e}\) being the zero matrix becomes negligible even for small parameters.

\subsubsection{Improvement by \texorpdfstring{\textcite{LB88}}{Lee and
    Brickell}}

As discussed in \cref{sec:ISD-H}, \textcite{LB88} published a variation of
\PISD{} (called \LBISD{} hereafter), by allowing \(p \in \{0, \ldots, w\}\)
errors within the information set.  These \(p\) errors are then decoded by
brute\-/force enumeration.  In the Hamming metric, this is equivalent to
iterating over all words in a Hamming sphere of radius \(p\).  However, in the
rank metric, we are presented with two options:
\begin{enumerate}
\item Iterate over rank\=/\(p\) matrices; or
\item Iterate over \(p\)\=/dimensional subspaces and solve a MinRank\-/like
  subproblem.
\end{enumerate}
We will see that the second option is the more favorable one.
\Cref{alg:LB-ISD-R-SC} shows the procedure.  Evidently, the case of \(p = 0\)
corresponds directly to \PISD\@.

\begin{algorithm}
  \caption{\LBISD{} using sets of coordinates in the rank
    metric}\label{alg:LB-ISD-R-SC}
  \DontPrintSemicolon{}%
  \LinesNumberedHidden{}%
  \KwIn{Generator matrix \(\mat{G} \in \FF_{q^m}^{k \times n}\), received word
    \(\vec{y} \in \FF_{q^m}^n\), target weight \(w \in \NN\), and parameter
    \(p \in \NN\) with \(p \le w\)}%
  \KwOut{Error \(\vec{e} \in \FF_{q^m}^n\) with \(\normR{\vec{e}}{q} \le w\)
    and \(\vec{y} - \vec{e} \in \mathcal{C}\)}%
  \Begin{%
    \Repeat{a solution is found}{%
      \(\mathcal{I} \sample \{1, \ldots, n\}\) with \(\card{\mathcal{I}} = k\)\;%
      \(\vec{e}' \gets
      \vec{y} - \vec{y}_{\mathcal{I}} \mat{G}_{\mathcal{I}}^{-1} \mat{G}\)\;%
      \ForAll{\(\mathcal{V} \le \FF_q^m\) with \(\dim \mathcal{V} = p\)}{%
        \(\basis{\vec{\beta}_1, \ldots, \vec{\beta}_p} \gets\) basis of
        \(\mathcal{V}\)\;%
        \(\vec{e}'' \coloneqq
        \begin{bmatrix} e''_1 & \cdots & e''_k \end{bmatrix}
        \mat{G}_{\mathcal{I}}^{-1} \mat{G}\)\;%
        \(e''_j \coloneqq \sum_{l = 1}^p a_{j, l} \vec{\beta}_l\)%
        \tcc*[r]{the \(a_{j, l}\)'s are unknowns}%
        \ShowLnLabel{alg:MR-subproblem}%
        \uIf{\(\exists(\)assignment to the
          \(a_{j, l}\)'s\()\ldotp \normR{\vec{e}' - \vec{e}''}{q} \le w\)}{%
          \Return{\((\vec{e}' - \vec{e}'')\) with the \(a_{j, l}\)'s
            substituted}%
        }%
      }%
    }%
  }
\end{algorithm}

Given a guess for \(\mathcal{V}\), the subproblem on \autoref{alg:MR-subproblem}
can be formulated as a MinRank instance using the idea outlined in the proof of
\cref{thm:RD-to-MR}.  The set of solutions to the subproblem is
\(\MR(k p, m, n, p, \FF_q; \vec{e}'; \mat{\Beta}_{1, 1}, \ldots, \mat{\Beta}_{k, p})\),
where
\[
  \mat{\Beta}_{j, l} \coloneqq
  \begin{bmatrix} \mat{0}_{m \times (k - j - 1)} & \vec{\beta}_l^{\trans} & \vec{0}_{m \times j} \end{bmatrix}\text.
\]

More formally, consider \(\vec{e}''\) written as a matrix over \(\FF_q\):
\begin{align*}
  \vec{e}'' &=
  \begin{bmatrix}
    \sum_{l = 1}^{p} a_{1, l} \vec{\beta}_l & \cdots & \sum_{l = 1}^{p} a_{k, l} \vec{\beta}_l
  \end{bmatrix}\\
            &= \sum_{j = 1}^k \sum_{l = 1}^p a_{j, l} \mat{\Beta}_{j, l}\text,
\end{align*}
with \(\mat{\Beta}_{j, l}\) as given above.  This, along with the fact that we
can take \(\vec{e}'\) as our \enquote{base} matrix (\(\mat{M}_0\) in
\cref{def:MR}) leads to the MinRank instance above.

The set of solutions can be retrieved in time
\(\mathcal{O}(k p {(\ceil{k p / {m}} m)}^2 q^{\ceil{k p / {m}} p})\) using the
kernel attack (\cref{sec:kernel-attack}).  In order to fully evaluate the
complexity of \cref{alg:LB-ISD-R-SC}, we need to determine the probability of a
successful iteration.  This probability is formally defined in
\cref{def:submatrix-rank-distribution}.

\begin{definition}[Probability of submatrix having given
  rank]\label{def:submatrix-rank-distribution}
  Let \(\mat{X} \in \FF_q^{m \times n}\) be a matrix with \(\rank \mat{X} = w\).
  Given a set of coordinates \(\mathcal{I} \subset \{1, \ldots, n\}\) with
  \(\card{\mathcal{I}} = k \le n\), what is the probability that
  \(\rank \mat{X}_{\mathcal{I}} = p\)?  We denote this probability by
  \[
    p_{\LB}(p; m, n, k, \FF_q, w)\text.
  \]
  \begin{remark}
    Note that a solution for \(p_{\LB}(p; m, n, k, \FF_q, w)\) would have to
    average over all possible inputs.  As an example, consider the two matrices
    \[
      \begin{bmatrix}
        1 & 0 & 0 & 1 \\
        0 & 1 & 0 & 1 \\
        0 & 0 & 1 & 1
      \end{bmatrix}\ \text{and}\
      \begin{bmatrix}
        1 & 0 & 0 & 0 \\
        0 & 1 & 0 & 0 \\
        0 & 0 & 1 & 0
      \end{bmatrix}\text.
    \]
    Both matrices are of rank \(3\) \--{} however, if one were to pick a
    \(3 \times 3\) submatrix at random, the probability that the submatrix is
    also of rank \(3\) is higher in the first case than in the second.  This is
    so because
    \(\begin{bsmallmatrix} 1 & 0 & 1 \\ 0 & 1 & 1 \\ 0 & 0 & 1 \end{bsmallmatrix}\),
    \(\begin{bsmallmatrix} 1 & 0 & 1 \\ 0 & 0 & 1 \\ 0 & 1 & 1 \end{bsmallmatrix}\),
    \(\begin{bsmallmatrix} 0 & 1 & 1 \\ 1 & 0 & 1 \\ 0 & 0 & 1 \end{bsmallmatrix}\),
    \(\begin{bsmallmatrix} 0 & 0 & 1 \\ 1 & 0 & 1 \\ 0 & 1 & 1 \end{bsmallmatrix}\),
    \(\begin{bsmallmatrix} 0 & 1 & 1 \\ 0 & 0 & 1 \\ 1 & 0 & 1 \end{bsmallmatrix}\),
    and
    \(\begin{bsmallmatrix} 0 & 0 & 1 \\ 0 & 1 & 1 \\ 1 & 0 & 1 \end{bsmallmatrix}\)
    are all rank\=/\(3\) submatrices of the first matrix, but are not
    submatrices at all of the second matrix.
  \end{remark}
\end{definition}

\Cref{tab:SMRD-empirical} shows the value of
\(p_{\LB}(p; 32, 32, 16, \FF_2, 8)\) for different values of \(p\).  The
probability was determined empirically by \num{6e8} simulations using
\textsc{SageMath}.  The source code can be seen in \cref{lst:SMRD-simulation}.

\begin{table}
  \centering \captionabove{\(p_{\LB}(p; 32, 32, 16, \FF_2, 8)\) for different
    values of \(p\)}\label{tab:SMRD-empirical}
  \begin{tabular}{%
    S[table-format=1]%
    S[table-format=1.2e-1, round-integer-to-decimal, table-auto-round]%
    }
    \toprule
    {\(p\)} & {\(p_{\LB}(p; 32, 32, 16, \FF_2, 8)\)} \\
    \midrule
    0 & 0 \\
    1 & 0 \\
    2 & 0 \\
    3 & 0 \\
    4 & 2.00000000000000e-8 \\
    5 & 4.12666666666667e-6 \\
    6 & 4.31678333333333e-4 \\
    7 & 0.0294282400000000 \\
    8 & 0.970135935000000 \\
    \bottomrule
  \end{tabular}
\end{table}

Knowing that the number of subspaces of dimension \(p\) in a vector space of
dimension \(m\) over \(\FF_q\) is given by
\[
  \gbinom{m}{p}_q =
  \begin{cases}
    \frac{%
      (1 - q^m)(1 - q^{m - 1}) \cdots (1 - q^{m - r + 1})%
    }{%
      (1 - q)(1 - q^2) \cdots (1 - q^r)%
    } & \text{if}\ p \le m \\
    0 & \text{if}\ p > m\text,
  \end{cases}
\]
we see from \cref{thm:las-vegas-running-time} that the complexity of
\cref{alg:LB-ISD-R-SC} satisfies
\begin{equation}\label{eq:LB-ISD-R-SC-complexity}
  \mathcal{O}\left(
    {p_{\LB}(p; m, n, k, \FF_q, w)}^{-1}
    \gbinom{m}{p}_q
    k p {(\ceil{k p / {m}} m)}^2 q^{\ceil{k p / {m}} p}
  \right)\text.
\end{equation}

Instead of solving a MinRank subproblem on \autoref{alg:MR-subproblem} by
iterating over all \(p\)\=/dimensional subspaces, one can iterate over all
weight\=/\(p\)\--- i.e., rank\=/\(p\)\--- matrices \(\vec{e}''\) and check if
\[
  \normR{\vec{e}' - \vec{e}'' \mat{G}_{\mathcal{I}}^{-1} \mat{G}}{q} \le w
\]
immediately.  In order to evaluate the complexity of such a variant of the
algorithm, we need \cref{thm:number-of-surjective-linear-transformations} and
\cref{cor:number-of-matrices-of-given-rank}.

\begin{theorem}\label{thm:number-of-surjective-linear-transformations}
  The number of surjective linear transformations from an \(n\)\=/dimensional
  vector space \(V_n(q)\) to an \(m\)\=/dimensional vector space over \(\FF_q\) is
  \[
    \sum_{k = 0}^m {(-1)}^{m - k} \gbinom{m}{k}_q q^{n k + \binom{m - k}{2}}\text.
  \]
  \begin{proof}
    See~\cite[p.~338]{LW01}.
  \end{proof}
\end{theorem}

\begin{corollary}\label{cor:number-of-matrices-of-given-rank}
  The number of \(k \times n\) matrices over \(\FF_q\) that have rank \(p\) is
  \[
    N_q(k, n, p) \coloneqq
    \gbinom{n}{p}_q \sum_{l = 0}^{p}
    {(-1)}^{r - l} \gbinom{r}{l}_q q^{k l + \binom{r - l}{2}}\text.
  \]
\end{corollary}

\Cref{cor:number-of-matrices-of-given-rank} implies that an algorithm which
iterates through all rank\=/\(p\) matrices would have a complexity of
\[
  \mathcal{O}\left(
    {p_{\LB}(p; m, n, k, \FF_q, w)}^{-1} N_q(m, k, p)
  \right)\text.
\]
However, we found out numerically that this quantity is always larger than that
in \cref{eq:LB-ISD-R-SC-complexity}.

\subsubsection{Thoughts on the Ideas by \texorpdfstring{\textcite{Ste89,
      Dum96}}{Stern and Dumer}}

As opposed to Lee\--Brickell's algorithm which decodes \(p\) errors in the
information set by brute force, Stern\--Dumer's algorithm partitions the
information set into two subsets of size \(k / {2}\) each, and allows for \(p\)
errors within either of them.  This has the effect of splitting the enumeration
into two parts of size \(k / {2}\) each.

However, the most notable improvement of Stern\--Dumer's algorithm is the
incorporation of the idea by \textcite{Leo88} to restrict the number of
possible candidates for the error word \(\vec{e}\) to those vectors having
\(l \in \{0, \ldots, n - k\}\) zeroes in positions outside of the
\(\mathcal{I}\)\=/indexed columns.  Thus, the main difference to
Lee\--Brickell's algorithm is that Stern\--Dumer's algorithm tries to build the
weight\=/\(w\) error word by first looking for collisions among sums of \(l\)
columns restricted to certain positions and then checks the weight of the
column sums arising from the collisions.

Unfortunately, we already noted that the probability of having zero entries is
overwhelmingly small when errors are chosen according to their rank weight
rather than Hamming weight.  This makes Stern\--Dumer's algorithm unsuitable
for the task, similarly to \PISD\@.

\subsection{Algorithms Using Projections in the Rank Metric}\label{sec:ISD-R-P}

In the Hamming metric, the notion of \emph{error support} is used to formally
define the probability of success of information set decoding algorithms.  We
saw that the error support, \(\supp(\vec{e})\), is given by the coordinates of
the nonzero entries in \(\vec{e}\), i.e., the entries contributing to the
overall Hamming weight.

It is a well\-/known fact from linear algebra that a vector
\(x \in \FF_{q^m}^n\) with \(\normR{\vec{x}}{q} \le p\) can be represented as
\(\vec{x} =
\begin{bmatrix} x_1 & \cdots & x_n \end{bmatrix} =
\begin{bmatrix} x'_1 & \cdots & x'_p \end{bmatrix} \mat{P}\), where
\(x'_j \in \FF_{q^m}\) and \(\mat{P}\) is a \(p \times n\) matrix over
\(\FF_q\) of full rank.  The concept of \emph{elementary linear subspaces} can
be introduced as a consequence of this representation.

\begin{definition}[Elementary linear subspace]
  A subspace \(\mathcal{V} \le \FF_{q^m}^n\) is said to be \emph{elementary} if
  it has a basis \(\Beta\) consisting of row vectors in \(\FF_q^n\).  \(\Beta\)
  is called an elementary basis of \(\mathcal{V}\).  For \(0 \le v \le n\), we
  define \(E_p(q^m, n)\) as the set of all ELS's with dimension \(p\) in
  \(\FF_{q^m}^n\).
\end{definition}

\Textcite{GY08} formally defined this concept and showed the connection between
\emph{sets of coordinates} in the Hamming metric and \emph{elementary linear
  subspaces} in the rank metric.  Following the notion of elementary linear
subspaces, we can define \(\supp(\vec{e})\) in the rank metric to be the
\(\FF_q\)\=/linear subspace of \(\FF_q^m\) spanned by the columns of
\(\vec{e}\).  Now, what we would like to do is project the columns of
\(\vec{e}\) onto a subspace that we hope would be orthogonal or
\enquote{near\-/orthogonal} to \(\supp(\vec{e})\).  This is highly desired
because projecting the problem onto such a subspace would have the effect of
eliminating \(\vec{e}\) from the picture.  The subsequent algorithms have this
idea at their core.

\subsubsection{Plain Information Set Decoding}

Informally, we are interested in choosing a subspace of \(\FF_q^m\) would:
\begin{conditions}
\item\label{cond:orthogonality} Be orthogonal or \enquote{near\-/orthogonal} to
  \(\supp(\vec{e})\); and
\item\label{cond:determinedness} Allow us to solve for the message
  \(\vec{x} \in \FF_{q^m}^k\).
\end{conditions}

We know that\--- in full generality\--- \(e_i \in \FF_q^m\) for all
\(1 \le i \le n\).  Let \(\mathcal{V}\) be a \(\zeta\)\=/dimensional subspace
of \(\FF_q^m\).  Now, assume that it is also true that
\(e_i \in \mathcal{V}^{\perp}\) for all \(1 \le i \le n\).  (In other words,
\(\mathcal{V}\) satisfies \cref{cond:orthogonality}.)  We can write
\begin{equation}\label{eq:P-ISD-R-P-system}
  \setlength{\arraycolsep}{0pt}
  \newcolumntype{B}{>{{}}c<{{}}}
  \left\{
    \begin{array}{l B l}
      \mat{P} (y_1 - {(\vec{x} \mat{G})}_1) & = & \vec{0} \\
      \mat{P} (y_2 - {(\vec{x} \mat{G})}_2) & = & \vec{0} \\
                                            & \vdots & \\
      \mat{P} (y_n - {(\vec{x} \mat{G})}_n) & = & \vec{0}\text,
    \end{array}
  \right.
\end{equation}
where \(\mat{P}\) is a \(\zeta \times m\) full\-/rank which projects from
\(\FF_q^m\) onto \(\mathcal{V}\), and \(\vec{x}\) is a \enquote{matrix} of
\(m k\) unknowns \--{}
\(x_{1, 1}, \ldots, x_{m, 1}, \ldots, x_{1, k}, \ldots, x_{m, k} \in \FF_{q}\).
When looked as a linear system over \(\FF_q\), \cref{eq:P-ISD-R-P-system} is a
system of \(\zeta n\) equations in \(m k\) unknowns.  Since we would like these
numbers to be as close as possible but avoid the case of having more unknowns
than equations, we take \(\zeta \coloneqq \ceil{m k / {n}}\).  In other words,
taking \(\zeta \coloneqq \ceil{m k / {n}}\) allows us to satisfy
\cref{cond:determinedness} simultaneously.

It is instructional to compare this approach to the kernel attack on MinRank
(\cref{sec:kernel-attack}).  In the kernel attack, we are given the
right\-/hand side of each equality of \cref{eq:kernel-attack} and are solving
for the left\-/hand side; here, we are given the left\-/hand side of each
equality of \cref{eq:P-ISD-R-P-system} and are solving for the right\-/hand
side.  We can translate this idea to an algorithm in a straightforward manner,
\cref{alg:P-ISD-R-P}.

\begin{algorithm}
  \caption{\PISD{} using projections in the rank metric}\label{alg:P-ISD-R-P}
  \DontPrintSemicolon{}%
  \LinesNumberedHidden{}%
  \KwIn{Generator matrix \(\mat{G} \in \FF_{q^m}^{k \times n}\), received word
    \(\vec{y} \in \FF_q^n\), and target weight \(w \in \NN\)}%
  \KwOut{Error \(\vec{e} \in \FF_q^n\) with \(\norm{\vec{e}} \le w\) and
    \(\vec{y} - \vec{e} \in \mathcal{C}\)}%
  \Begin{%
    \(\zeta \gets \ceil*{\frac{m k}{n}}\)\;%
    \(\vec{x} \coloneqq \begin{bmatrix} x_1 & \cdots & x_k \end{bmatrix}\) with
    \(x_j \in \FF_q^m\)\;%
    \(x_j \coloneqq
    \begin{bmatrix} x_{j, 1} & \cdots & x_{j, m} \end{bmatrix}^{\trans}\)
    with \(x_{j, l} \in \FF_q\)\;%
    \ShowLnLabel{alg:number-of-iterations}%
    \Repeat{sufficiently many iterations have been performed}{%
      \(\mathcal{V} \gets\) random \(\zeta\)\=/dimensional subspace of
      \(\FF_q^m\)\;%
      \(\mat{P} \gets\) \(\zeta \times m\) matrix that projects from
      \(\FF_q^m\) to \(\mathcal{V}\)\;%
      solve for \(\vec{x}\) in
      \(\{\mat{P} (y_i - {(\vec{x} \mat{G})}_i) = \vec{0} : \forall 1 \le i \le
      n\}\)\;%
      \ForAll{solutions for \(\vec{x}\)}{%
        \(\vec{e} \gets \vec{y} - \vec{x} \mat{G}\) with the appropriate values
        substituted in for \(\vec{x}\)\;%
        \lIf{\(\normR{\vec{e}}{q} \le w\)}{\Return{\(\vec{e}\)}%
        }%
      }%
    }%
  }
\end{algorithm}

Analyzing the complexity of \cref{alg:P-ISD-R-P} requires us to define what
\enquote{sufficiently many iterations} on \autoref{alg:number-of-iterations}
means.  We know that there are \(\gbinom{m}{w}_q\) possibilities for
\(\supp(\vec{e})\) in the general case.  On the other hand, there are
\(\gbinom{m - \zeta}{w}_q\) subspaces satisfying \cref{cond:orthogonality}.
(Recall that \cref{cond:determinedness} is satisfied by the choice of
\(\zeta\).)  This implies that the probability of a successful iteration is
\[
  \gbinom{m - \zeta}{w}_q \gbinom{m}{w}_q^{-1}\text;
\]
which, in turn, means that on average
\[
  \gbinom{m - \zeta}{w}_q^{-1} \gbinom{m}{w}_q
\]
iterations have to be performed (\cref{thm:las-vegas-running-time}).  Solving
for \(\vec{x}\) can be done in time polynomial in the inputs to the algorithm,
as it is a simple linear system over \(\FF_q\).

\subsubsection{The Lee\--Brickell Algorithm}

Bearing in mind the notion of \(\supp(\vec{e})\), we can adapt the idea of
\textcite{LB88} from the Hamming metric to the rank metric.  Informally, we can
allow for \(\supp(\vec{e})\) and \(\mathcal{V}\) to be
\enquote{near\-/orthogonal}, i.e., for a \(p\)\=/dimensional intersection
between them.  The dimension of the intersection is determined by a parameter
\(p \in \NN\) with \(0 \le p \le w\).  \PISD{} corresponds directly to
\(p = 0\).

Allowing for a nontrivial intersection has the effect of increasing the
probability of a successful iteration.  Namely, let
\(\supp(\vec{e}) \cap \mathcal{V} = \tilde{\mathcal{V}}\), with
\(\tilde{\mathcal{V}}\) being a \(p\)\=/dimensional subspace of
\(\mathcal{V}\).  Now, \enquote{\(p\) dimensions} of \(\supp(\vec{e})\) come
from \(\tilde{\mathcal{V}}\), while the remaining \enquote{\(w - p\)
  dimensions} come from \(\mathcal{V}^{\perp}\).  Again, there are in total
\(\gbinom{m}{w}_q\) possibilities for \(\supp(\vec{e})\).  This means that the
probability of a successful iteration is now
\[
  \gbinom{\zeta}{p}_q \gbinom{m - \zeta}{w - p}_q \gbinom{m}{w}_q^{-1}\text.
\]

Let \(\basis{\tilde{\vec{v}}_1, \ldots, \tilde{\vec{v}}_p}\) be a basis of
\(\tilde{\mathcal{V}}\).  \Cref{eq:P-ISD-R-P-system} now becomes
\begin{equation}\label{eq:LB-ISD-R-P-system}
  \setlength{\arraycolsep}{0pt}
  \newcolumntype{B}{>{{}}c<{{}}}
  \left\{
    \begin{array}{l B l}
      \mat{P} (y_1 - {(\vec{x} \mat{G})}_1)
      & = &
            \sum_{j = 1}^p \lambda_{1, j} \tilde{\vec{v}}_j \\
      \mat{P} (y_2 - {(\vec{x} \mat{G})}_2)
      & = &
            \sum_{j = 1}^p \lambda_{2, j} \tilde{\vec{v}}_j \\
      & \vdots &                                          \\
      \mat{P} (y_n - {(\vec{x} \mat{G})}_n)
      & = &
            \sum_{j = 1}^p \lambda_{n, j} \tilde{\vec{v}}_j\text.
    \end{array}
  \right.
\end{equation}
There are an additional \(n p\) unknowns now; thus, we require that
\(\zeta = \ceil{m k / {n}} + p\).  \Cref{alg:LB-ISD-R-P} is an immediate
consequence of \cref{eq:LB-ISD-R-P-system}.  \Cref{tab:LB-ISD-R-P-complexity}
shows the complexity for different values of \(p\) for the parameters proposed
for the Loidreau cryptosystem.

\begin{algorithm}
  \caption{\LBISD{} using projections in the rank metric}\label{alg:LB-ISD-R-P}
  \DontPrintSemicolon{}%
  \KwIn{Generator matrix \(\mat{G} \in \FF_{q^m}^{k \times n}\), received word
    \(\vec{y} \in \FF_q^n\), target weight \(w \in \NN\), and algorithm
    parameter \(p \in \NN\) with \(0 \le p \le w\)}%
  \KwOut{Error \(\vec{e} \in \FF_q^n\) with \(\norm{\vec{e}} \le w\) and
    \(\vec{y} - \vec{e} \in \mathcal{C}\)}%
  \Begin{%
    \(\zeta \gets \ceil*{\frac{m k}{n}} + p\)\;%
    \(\vec{x} \coloneqq \begin{bmatrix} x_1 & \cdots & x_k \end{bmatrix}\) with
    \(x_j \in \FF_q^m\)\;%
    \(x_j \coloneqq \begin{bmatrix} x_{j, 1} & \cdots & x_{j, m} \end{bmatrix}^{\trans}\)
    with \(x_{j, l} \in \FF_q\)\;%
    \Repeat{sufficiently many iterations have been performed}{%
      \(\mathcal{V} \gets\) random \(\zeta\)\=/dimensional subspace of
      \(\FF_q^m\)\;%
      \(\mat{P} \gets\) \(\zeta \times m\) matrix that projects from
      \(\FF_q^m\) to \(\mathcal{V}\)\;%
      \ForEach{\(\tilde{\mathcal{V}} \le \mathcal{V}\) with
        \(\dim \tilde{\mathcal{V}} = p\)}{%
        \(\basis{\tilde{\vec{v}}_1, \ldots, \tilde{\vec{v}}_p} \gets\) basis of
        \(\tilde{\mathcal{V}}\)\;%
        solve for \(\vec{x}\) and \(\{\lambda_{i, j}\}\) in
        \(\{\mat{P} (y_i - {(\vec{x} \mat{G})}_i) =
        \sum_{j = 1}^p \lambda_{i, j} \tilde{\vec{v}}_j : \forall 1 \le i \le n\}\)\;%
        \ForAll{solutions for \(\vec{x}\) and \(\lambda_{i, j}\)}{%
          \(\vec{e} \gets \vec{y} - \vec{x} \mat{G}\) with the appropriate
          values for \(\vec{x}\) substituted in\;%
          \lIf{\(\norm{\vec{e}}_q \le w\)}{\Return{\(\vec{e}\)}%
          }%
        }%
      }%
    }%
  }
\end{algorithm}

\paragraph{Asymptotic analysis of \cref{alg:LB-ISD-R-P}} Recall that the
Gaussian binomial coefficient satisfies
\[
  \gbinom{n}{k}_q \in \Theta(q^{k (n - k)})\text.
\]
This means that\--- ignoring the cost of Gaussian elimination\--- the
complexity of \cref{alg:LB-ISD-R-P} is
\[
  \Theta(q^{w (\ceil{m k / {n}} + p) + p m - 2p (\ceil{m k / {n}} + 2p)})\text.
\]
The optimal value for \(p\) depends on the instance's parameters, but the
optimal value can easily be computed as the value which minimizes the
complexity.  We observed that is beneficial to either leave \(p = 0\) or push
it as close to \(w\) as possible (usually exactly to \(w\)).  When \(p = 0\),
the above expression becomes
\[
  \Theta(q^{w \ceil{m k / {n}}})\text,
\]
which can be seen as a dual to the state\-/of\-/the\-/art algorithms by
\textcites{GRS13}{HT15}.  We see that \LBISD{} has the same asymptotic
complexity as the algorithms in~\cites{GRS13}{HT15} in the case when \(p = 0\)
is the optimal value, and considerably better complexity otherwise.

\begin{table}
  \centering
  \captionabove{Complexity of \LBISD{} for the parameters proposed
    in~\cite{Loi17}}\label{tab:LB-ISD-R-P-complexity}
  \begin{tabular}{%
    S[table-format=3]%
    S[table-format=3]%
    S[table-format=2]%
    S[table-format=1]%
    S[table-format=3]%
    S[table-format=1]%
    S[table-format=3]%
    }
    \toprule
    {\(m\)} & {\(n\)} & {\(k\)} & {\(\lambda\)}
    & {Compl.\ of~\cite{GRS13} (\si{bit})}
    & {Opt.\ \(p\)} & {Compl.\ of \cref{alg:LB-ISD-R-P} (\si{bit})} \\
    \midrule
    50 & 50 & 32 & 3 & 96 & 3 & 36 \\
    80 & 80 & 11 & 3 & 121 & 0 & 121 \\
    96 & 64 & 40 & 3 & 240 & 4 & 112 \\
    128 & 90 & 24 & 3 & 352 & 0 & 352 \\
    128 & 120 & 80 & 5 & 320 & 4 & 128 \\
    \bottomrule
  \end{tabular}
\end{table}

\chapter{Conclusion and Future Perspective}

In this thesis, we surveyed the GPT cryptosystem and proposed new techniques
for decoding in the rank metric.  These techniques are based on the concept of
\emph{information set decoding}, which has been studied extensively in the
Hamming metric.  For certain parameter sets, the newly\-/proposed techniques
were found to improve upon the current state\-/of\-/the\-/art.

In \cref{chap:GPT}, we defined the GPT cryptosystem in its original form and
the subsequent reparation by \textcite{GO01}.  We simplified and incorporated
additional ideas into the proof of \textcite{Ksh07} which shows that we can
always assume a certain form of a GPT public key suitable for cryptanalysis.

The main contributions are the algorithms in \cref{chap:ISD}.  We formulated
variants of plain information set decoding and Lee\--Brickell's algorithm using
a notion of error support suitable for the rank metric.  Prior to that, we
surveyed the connection between the MinRank problem in linear algebra and
decoding in the rank metric.  Additionally, the direct adaptation of
information set decoding from the Hamming metric to the rank metric was
experimentally found to be inefficient.  However, it served as the basis for
the algorithms in \cref{sec:ISD-R-P}, which are tailored specifically for the
rank metric and serve as a dual to the algorithm by \textcite{GRS13}.

We leave it as future work to
\begin{itemize}
\item Adapt improvements in the spirit of \textcite{Ste89, Dum96} to the notion
  of error support in the rank metric; and
\item Provide working implementations of algorithms in \textsc{SageMath}.
\end{itemize}

\appendix\chapter{Source code}

The code shown in \cref{lst:SMRD-simulation} estimates the empirical
distribution of \(p_{\LB}\) defined in \cref{def:submatrix-rank-distribution}.

\begin{listing}[h]
\begin{minted}[escapeinside=||, mathescape]{python}
from collections import Counter

def submatrix_rank_simulation(m, n, k, KK, w,
                              num_outer=10^4, num_inner=10^4):
    #  Simulate the distribution of the rank of a random submatrix.

    #  `num_outer` matrices from $\M(\KK, m, n)$ of rank $w$ are sampled.
    #  For each rank-$w$ matrix, `num_outer` $m \times k$ random submatrices
    #  are taken and have their ranks tallied.
    if k > n:
        raise ValueError("`k` must not be greater than `n`")

    tally = Counter()
    col_indices = tuple(range(n))

    for i in range(num_outer):
        E = random_matrix(KK, m, n,
                          algorithm='echelonizable', rank=w)
        for _ in range(num_inner):
            E_sub = E[:, sample(col_indices, k)]
            tally[E_sub.rank()] += 1

    return tally
\end{minted}
\captionbelow{Estimating \(p_{\LB}(p; 32, 32, 16, \FF_2, 8)\) in
  \textsc{SageMath}}\label{lst:SMRD-simulation}
\end{listing}

\backmatter\printbibliography{}

\end{document}

%%% Local Variables:
%%% mode: latex
%%% TeX-command-extra-options: "-shell-escape"
%%% TeX-engine: luatex
%%% TeX-master: t
%%% End:
